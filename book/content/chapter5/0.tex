CMake中,整个应用程序可以从单个源代码文件(如经典的helloworld.cpp)构建。但同样也可以创建一个项目,其中可执行文件是由许多源文件构建的:几十个甚至几千个。许多初学者都是这样做的:用几个文件构建二进制文件,让项目在没有严格计划的情况下自然增长。根据需要不断添加文件,所有东西都已经直接链接到一个没有结构的单一二进制文件中。

作为软件开发者,我们故意划界限,并将组件指定为将一个或多个翻译单元(.cpp文件)分组的部分,是为了提高代码的可读性,管理耦合和关联性,加快构建过程,并最终发现和提取可重用组件成为自治单元。

每个大型项目都会引入某种形式的划分,这里就是CMake目标发挥作用的地方。CMake目标代表了一个专注于特定目标的逻辑单元。目标可以依赖于其他目标,构建遵循声明方法。CMake负责确定构建目标的正确顺序,尽可能优化并行构建,并相应地执行必要步骤。作为一个通用原则,当一个目标构建时,会生成一个工件,该工件可以作为其他目标使用,或作为构建过程的最终输出。

注意“工件”这个词,我故意避免使用特定术语,因为CMake在生成可执行文件或库之外提供了灵活性。实际上,可以利用生成的构建系统来产生各种类型的输出:额外的源文件、头文件、目标文件、存档、配置文件等。唯一的要求是一个命令行工具(如编译器)、可选的输入文件,以及指定的输出路径。

目标是极其强大的概念,极大地简化了构建项目的过程。理解其如何工作,并掌握以优雅和有组织的方式配置技巧,也至关重要。这些知识确保了顺畅和高效的开发体验。

本章中,包含以下内容:

\begin{itemize}
\item
理解目标

\item
设置目标的属性

\item
编写自定义命令
\end{itemize}
















































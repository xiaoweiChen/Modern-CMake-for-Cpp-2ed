理解目标是编写清晰、现代的CMake项目的关键。在本章中,我们不仅讨论了什么是目标以及如何定义三种不同类型的目标:可执行文件、库和自定义目标。我们还解释了目标如何通过依赖关系图相互依赖,并学习了如何使用Graphviz模块可视化它。有了这种基本的理解,我们能够了解目标的关键特性——属性。我们不仅介绍了一些在目标上设置常规属性的命令,还解决了传播属性,也称为传递使用要求的谜团。

这是一个难以攻克的问题,因为我们需要理解的不仅是如何控制哪些属性被传播,还有这种传播如何影响后续的目标。此外,我们还发现了如何确保从多个来源消费的属性的兼容性。

然后,我们简要讨论了伪目标:导入的目标、别名目标和接口库。所有这些在以后的项目中都会派上用场,特别是当我们知道如何将它们与传播属性连接起来以获取利益时。接着,我们讨论了生成的构建目标以及配置阶段如何影响它们。之后,我们花了一些时间研究一种与目标相似但又不完全是目标的机制:自定义命令。我们提到了它们如何生成被其他目标(编译、翻译等)消费的文件,以及它们的钩子功能:在构建目标时执行额外步骤。

有了如此坚实的基础,我们准备好进行下一个主题——将C++源代码编译成可执行文件和库。
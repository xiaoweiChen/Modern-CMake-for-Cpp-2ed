
使用自定义目标有一个缺点——将它们添加到ALL目标或者依赖它们来构建其他目标,每次都会构建。有时候,这正是想要的,但需要自定义行为来生成不应该无故重新创建的文件:

\begin{itemize}
\item
生成另一个目标所依赖的源代码文件

\item
将另一种语言翻译成C++

\item
在另一个目标构建之前或之后,立即执行自定义操作
\end{itemize}

自定义命令有两个签名。第一个是add\_custom\_target()的扩展版本:

\begin{shell}
add_custom_command(OUTPUT output1 [output2 ...]
                   COMMAND command1 [ARGS] [args1...]
                   [COMMAND command2 [ARGS] [args2...] ...]
                   [MAIN_DEPENDENCY depend]
                   [DEPENDS [depends...]]
                   [BYPRODUCTS [files...]]
                   [IMPLICIT_DEPENDS <lang1> depend1
                                    [<lang2> depend2] ...]
                   [WORKING_DIRECTORY dir]
                   [COMMENT comment]
                   [DEPFILE depfile]
                   [JOB_POOL job_pool]
                   [VERBATIM] [APPEND] [USES_TERMINAL]
                   [COMMAND_EXPAND_LISTS])
\end{shell}

自定义命令并不创建逻辑目标,但与自定义目标一样,必须添加到依赖关系图中。有两种方法可以实现——将其输出工件作为可执行文件(或库)的源,或者显式地将其添加到自定义目标的DEPENDS列表中。

\mySubsubsection{5.3.1.}{将自定义命令用作生成器}

诚然,并非每个项目都需要从其他文件生成C++代码。这样的情况可能是编译Google的Protocol Buffer(Protobuf)的.proto文件。如果不熟悉这个库,Protobuf是一个用于结构化数据的平台无关的二进制序列化器。

可以用来将对象编码和解码为二进制流:文件或网络连接。为了保持Protobuf跨平台并且在同时保持快速,Google的工程师们发明了自己的Protobuf语言,在.proto文件中定义模型,如下所示:

\begin{shell}
message Person {
    required string name = 1;
    required int32 id = 2;
    optional string email = 3;
}
\end{shell}

这样的文件可以用来在多种语言中编码数据——C++、Ruby、Go、Python、Java等等。Google提供了编译器——protoc,读取.proto文件并为所选语言输出有效的结构和序列化源代码(稍后需要编译或解释)。聪明的工程师不会将这些生成的源文件检入仓库,而是使用原始的Protobuf格式并在构建链中添加一个步骤来生成源文件。

我们尚不知道如何检测目标主机上是否(以及在哪里)可用Protobuf编译器(我们将在第9章中介绍)。现在,假设编译器的protoc命令位于系统已知的某个位置,已经准备好了person.proto文件,并且知道Protobuf编译器将输出person.pb.h和person.pb.cc文件。以下是如何定义一个自定义命令来编译它们的例子:

\begin{cmake}
add_custom_command(OUTPUT person.pb.h person.pb.cc
        COMMAND protoc ARGS person.proto
        DEPENDS person.proto
)
\end{cmake}

然后,为了允许可执行文件进行序列化,可以简单地将输出文件添加到源文件中:

\begin{cmake}
add_executable(serializer serializer.cpp person.pb.cc)
\end{cmake}

假设正确处理了头文件的包含和Protobuf库的链接,当对.proto文件进行更改时,一切都会自动编译和更新。

一个简化的(实用性要小得多)示例是通过从另一个位置复制来创建必要的头文件:

\filename{ch05/03-command/CMakeLists.txt}

\begin{cmake}
add_executable(main main.cpp constants.h)
target_include_directories(main PRIVATE ${CMAKE_BINARY_DIR})
add_custom_command(OUTPUT constants.h COMMAND cp
                   ARGS "${CMAKE_SOURCE_DIR}/template.xyz" constants.h)
\end{cmake}

这时,“编译器”是cp命令。它通过从源树中复制到构建树根目录,创建一个constants.h文件,从而满足main目标的依赖。

\mySubsubsection{5.3.2.}{将自定义命令用作目标钩子}

add\_custom\_command()命令的第二个版本引入了一种机制,用于在构建目标之前或之后执行命令:

\begin{cmake}
add_custom_command(TARGET <target>
                   PRE_BUILD | PRE_LINK | POST_BUILD
                   COMMAND command1 [ARGS] [args1...]
                   [COMMAND command2 [ARGS] [args2...] ...]
                   [BYPRODUCTS [files...]]
                   [WORKING_DIRECTORY dir]
                   [COMMENT comment]
                   [VERBATIM] [USES_TERMINAL]
                   [COMMAND_EXPAND_LISTS])
\end{cmake}

通过第一个参数指定想要用新行为“增强”的目标,并在满足以下条件的情况下执行:

\begin{itemize}
\item
PRE\_BUILD 将在此目标的所有其他规则之前运行(仅限Visual Studio生成器;对于其他生成器,行为类似于PRE\_LINK)。

\item
PRE\_LINK 将命令绑定在所有源代码编译完成后,但在链接(或归档)目标之前运行。不适用于自定义目标。

\item
POST\_BUILD 将在此目标的所有其他规则执行完毕后运行。
\end{itemize}

使用这个版本的add\_custom\_command(),可以复现之前BankApp示例中的校验和生成:

\filename{ch05/04-command/CMakeLists.txt}

\begin{cmake}
cmake_minimum_required(VERSION 3.26)
project(Command CXX)
add_executable(main main.cpp)
add_custom_command(TARGET main POST_BUILD
                   COMMAND cksum
                   ARGS "$<TARGET_FILE:main>" > "main.ck")
\end{cmake}

主可执行文件的构建完成后,CMake将使用提供的参数执行cksum。但是第一个参数中发生了什么?它不是一个变量,如果是变量,它会被大括号(\$\{\})包裹,而不是尖括号(\$<>)。它是一个生成器表达式,计算结果为目标二进制文件的完整路径。这种机制在许多目标属性的上下文中非常有用,我们将在下一章中详细解释。



























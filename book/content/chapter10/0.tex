C++20 引入了一个新特性:模块。它们用模块文件替代了头文件中的纯文本符号声明,该模块文件将预编译为中间二进制格式,大大减少了构建时间。

我们将讨论 CMake 中 C++20 模块的最重要主题,从 C++20 模块作为一个概念的一般介绍开始:它们相对于标准头文件的优点以及它们如何简化源代码中单元的管理。尽管简化构建过程的承诺令人兴奋,但本章强调了它们被采纳的道路既困难又漫长。

理论部分结束后,我们将继续讨论在项目中实现模块的实际方面:我们将讨论在早期版本的 CMake 中启用它们的实验性支持,以及在 CMake 3.28 中的完整发布。

我们通过 C++20 模块的旅程不仅仅是为了理解一个新特性 —— 它是关于重新思考在大型 C++ 项目中组件如何交互。在本章结束时,你不仅会掌握模块的理论方面,还能通过示例获得实践经验,增强利用这一特性实现更好项目成果的能力。

在本章中,将包含以下内容:

\begin{itemize}
\item
C++20 模块是什么?

\item
使用 C++20 模块支持的编写项目

\item
配置工具链
\end{itemize}















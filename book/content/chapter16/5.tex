Workflow presets are the ultimate automation solution for our project. They allow us to automatically execute multiple stage-specific presets in the predetermined order. That way, we can practically perform an end-to-end build in a single step.

To discover available workflows for a project, we can execute the following command:

\begin{shell}
cmake --workflow --list-presets
\end{shell}

To select and apply a preset, use the following command:

\begin{shell}
cmake –workflow --preset <preset-name>
\end{shell}

Additionally, with the -{}-fresh flag, we can wipe the build tree and clear the cache.

Defining workflow presets is quite simple; we need to define a name and we can optionally provide displayName and description, just like for stage-specific presets. After that, we must enumerate all the stage-specific presets the workflow should execute. This is done by providing a steps array containing objects with type and name properties, as illustrated here:

\filename{ch16/01-presets/CMakePresets.json (continued)}

\begin{json}
...
"workflowPresets": [
{
    "name": "myWorkflow",
    "steps": [
        {
            "type": "configure",
            "name": "myConfigure"
        },
        {
            "type": "build",
            "name": "myBuild"
        },
        {
            "type": "test",
            "name": "myTest"
        },
        {
            "type": "package",
            "name": "myPackage"
        },
        {
            "type": "build",
            "name": "myInstall"
        }
    ]
...
\end{json}

Each object in the steps array references a preset we defined earlier in this chapter, indicating its type (configure, build, test, or package) and a name. These presets collectively execute all necessary steps to fully build and install a project from scratch with a single command:

\begin{shell}
cmake --workflow --preset myWorkflow
\end{shell}

Workflow presets are the ultimate solution for automating C++ building, testing, packaging, and installing. Next, let’s explore how to manage some edge cases with conditions and macros.





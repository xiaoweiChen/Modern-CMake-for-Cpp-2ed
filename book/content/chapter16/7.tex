We have just completed a comprehensive overview of CMake presets, introduced in CMake 3.19, to streamline project management. Presets allow product authors to provide a neatly prepared experience for their users by configuring all the stages of the project build and delivery. Presets not only simplify the usage of CMake but also enhance consistency and allow environment-aware setups.

We explained the structure and usage of the CMakePresets.json and CMakeUserPresets.json files, providing insights into defining various types of presets, such as configure presets, build presets, test presets, package presets, and workflow presets. Each type is described in detail: we learned about common fields, how to structure presets internally, establish inheritance between them, and the specific configuration options available for the end user.

For the configure preset, we covered important topics like selecting the generator, build, and installation directory, and linking presets together with the configurePreset field. We now know how to handle build presets and set the build job count, targets, and cleaning options. Then, we learned how the test preset assists with test selection through extensive filtering and ordering options, output formatting, and execution parameters such as timeouts and fault tolerance. We understand how to manage package presets by specifying package generators, filtering, and package metadata. We even introduced a workaround to execute the installation stage through a specialized build preset application.

Next, we discovered how workflow presets allow the grouping of multiple stage-specific presets.Finally, we discussed conditions and macro expressions, providing project authors with greater control over the behavior of individual presets and their integration into a workflow.

Our CMake journey is complete! Congratulations – you now possess all the tools necessary to develop, test, and package high-quality C++ software. The best way forward is to apply what you’ve learned and create excellent software for your users. Good luck!
CMake searches for CMakePresets.json and CMakeUserPresets.json in the top-level directory. Both files use the same JSON structure to define presets, hence there isn’t much difference between them to discuss. The format is a JSON object with the following keys:

\begin{itemize}
\item
version: This is a required integer that specifies the version of the preset JSON schema

\item
cmakeMinimumRequired: This is an object that specifies the required CMake version

\item
include: This is an array of strings that includes external presets from file paths provided in the array (since schema version 4)

\item
configurePresets: This is an array of objects that defines the configuration stage presets

\item
buildPresets: This is an array of objects that defines the build stage presets

\item
testPresets: This is an array of objects that are specific to the test stage presets

\item
packagePresets: This is an array of objects that are specific to the package stage presets

\item
workflowPresets: This is an array of objects that are specific to the workflow mode presets

\item
vendor: This is an object that contains custom settings defined by IDEs and other vendors; CMake does not process this field
\end{itemize}

When writing a preset, CMake requires the version entry to be present; other values are optional. Here’s an example preset file (actual presets will be added in subsequent sections):

\filename{ch16/01-presets/CMakePresets.json}

\begin{json}
{
    "version": 6,
    "cmakeMinimumRequired": {
        "major": 3,
        "minor": 26,
        "patch": 0
    },
    "include": [],
    "configurePresets": [],
    "buildPresets": [],
    "testPresets": [],
    "packagePresets": [],
    "workflowPresets": [],
    "vendor": {
        "data": "IDE-specific information"
    }
}
\end{json}

There’s no requirement to add empty arrays like in the preceding example; entries other than version are optional. Speaking of which, the appropriate schema version for CMake 3.26 is 6.

Now that we understand the structure of the preset file, let’s actually learn how to define the presets themselves.

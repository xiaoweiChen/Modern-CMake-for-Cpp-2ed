当讨论每个特定阶段预设的一般字段时,提到了 condition 字段。现在是回到这个话题的时候了。condition 字段可以启用或禁用一个预设,在与工作流集成时揭示其真正的潜力,它允许在某些条件下绕过不适合的预设,并创建替代的预设。

条件要求预设架构版本 3 或以上(在 CMake 3.22 中引入),并且是 JSON 对象,编码了一些简单的逻辑操作,可以根据所使用的操作系统、环境变量甚至选定的生成器等条件来判断预设是否适用。CMake 通过宏提供了这些数据,这些宏基本上是一组只读变量,可以在预设文件中使用。

condition 对象的结构根据检查类型的不同而变化。每个条件都必须包含一个 type 字段,以及由该类型定义的其他字段。其基本类型包括:

\begin{itemize}
\item
const: 这个检查 value 字段中提供的值是否为布尔真。

\item
equals 和 notEquals: 比较 lhs 字段的值与 rhs 字段的值。

\item
inList 和 notInList: 这些检查 string 字段中提供的值是否存在于 list 字段的数组中。

\item
matches 和 notMatches: 这些判断 string 字段的值是否符合 regex 字段中定义的模式。
\end{itemize}

一个示例条件如下所示:

\begin{json}
"condition": {
    "type": "equals",
    "lhs": "${hostSystemName}",
    "rhs": "Windows"
}
\end{json}

const 条件的实际用途,主要是为了在不从 JSON 文件中删除预设的情况下禁用。除了 const 之外,所有基本条件都允许在其引入的字段 (lhs, rhs, string, list, regex) 中使用宏。

高级条件类型,类似于“not”,“and”和“or”操作,使用其他条件作为参数:

\begin{itemize}
\item
not: 这是对 condition 字段中提供的条件进行布尔反转。

\item
anyOf 和 allOf: 这些检查 conditions 数组中的任意条件或所有条件是否为真。
\end{itemize}

例如:

\begin{json}
"condition": {
    "type": "anyOf",
    "conditions": [
        {
            "type": "equals",
            "lhs": "${hostSystemName}",
            "rhs": "Windows"
        },
        {
            "type": "equals",
            "lhs": "${hostSystemName}",
            "rhs": "Linux"
        }
    ]
}
\end{json}

这个条件在系统为 Linux 或 Windows 时计算为真。

通过这些示例,引入了第一个宏:\$\{hostSystemName\}。宏遵循简单的语法,并限于特定实例:

\begin{itemize}
\item
\$\{sourceDir\}: 源代码树的路径。

\item
\$\{sourceParentDir\}: 源代码树父目录的路径。

\item
\$\{sourceDirName\}: 项目的目录名。

\item
\$\{presetName\}: 预设的名称。

\item
\$\{generator\}: 用于创建构建系统的生成器。

\item
\$\{hostSystemName\}: 系统名称:Linux, Windows, 或 macOS 上的 Darwin。

\item
\$\{fileDir\}: 包含当前预设的文件名(当使用 include 数组导入外部预设时适用)。

\item
\$\{dollar\}: 转义的美元符号 \$。

\item
\$\{pathListSep\}: 、环境特定的路径分隔符。

\item
\$env\{<variable-name>\}: 如果预设指定了环境变量,则返回该环境变量(区分大小写),否则返回父环境中的值。

\item
\$penv\{<variable-name>\}: 返回父环境中的环境变量。

\item
\$vendor\{<macro-name>\}: 允许 IDE 厂商引入自己的宏。
\end{itemize}

这些宏为在预设及其条件中的使用提供了足够的灵活性,使我们能够根据需要有效地切换工作流步骤。



























我们的项目已经构建、测试并记录在案。现在,到了将其发布给用户的时候。本章主要关注需要采取的最后两个步骤:安装和打包。这些是我们迄今为止所学的一切基础上的高级技术:管理目标和它们的依赖关系,短暂的使用需求,生成器表达式等。

安装使得项目能够在整个系统中发现和访问。我们将介绍如何在不进行安装的情况下导出目标,以供其他项目使用;以及如何安装项目,以便于整个系统轻松访问。将了解如何配置项目,以自动将各种工件类型放置到适当的目录中。为了处理更高级的场景,将介绍用于安装文件和目录,以及执行自定义脚本和 CMake 命令的低层命令。

接下来,将探索设置可重用的 CMake 包,其他项目可以使用 find\_package() 命令来发现它们。我们将解释如何确保目标,和定义特定文件的系统位置。我们还将讨论如何编写基本和高级的配置文件,以及与包相关联的版本文件。然后,为了模块化,我们将简要介绍组件的概念,这既适用于 CMake 包也适用于 install() 命令。所有这些准备工作将为本章最后要介绍的方面铺平道路:使用 CPack 生成各种操作系统中的包管理器都能识别的存档、安装程序、捆绑包和包。这些包可以分发预构建的工件、可执行文件和库。这是最终用户开始使用软件的最简单方式。

本章中,将包含以下内容:

\begin{itemize}
\item
无需安装即可导出

\item
在系统上安装项目

\item
创建可重用包

\item
定义组件

\item
使用 CPack 打包
\end{itemize}









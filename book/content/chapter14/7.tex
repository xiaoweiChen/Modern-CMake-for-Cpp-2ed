编写跨平台安装脚本的复杂性可能会令人望而却步,但 CMake 显著简化了这一任务。尽管它需要一些初始设置,CMake 流程化了这一过程,与这本书中探讨的概念和技术无缝集成。

我们从理解如何从项目中导出 CMake 目标开始,使得它们可以在不需要安装的情况下在其他项目中使用。接下来,深入了解已经为导出配置的项目的安装。探讨安装基础时,专注于一个关键方面:安装 CMake 目标。现在掌握了 CMake 如何为不同工件类型分配不同目的地,以及公共头文件的特殊考虑。我们还检查了 install() 命令的其他模式,包括安装文件、程序和目录,以及在安装过程中执行脚本。

然后,CMake可创建重用包。我们探索了如何使项目目标可重定位,从而方便用户定义安装位置。这包括创建可以通过 find\_package() 使用的完全定义的包,涉及准备目标导出文件、配置文件和版本文件。考虑到不同用户的需求,了解了如何将工件和操作分组到安装组件中,将它们与 CMake 包的组件区分开来。我们的探索最终以 CPack 的介绍达到尾声,我们了解了如何准备基本的二进制包,提供了一种有效的方法来分发预编译的软件。虽然掌握 CMake 中安装和打包的细微之处是一个持续的过程,但这一章为我们奠定了坚实的基础。它使我们能够处理常见场景,并且自信地进一步深入。

下一章中,我们将运用我们积累的知识,通过创建一个连贯的、专业级别的项目,展示这些 CMake 技术的实际应用。
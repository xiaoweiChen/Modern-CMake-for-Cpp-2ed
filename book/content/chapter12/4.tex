
Valgrind(\url{https://www.valgrind.org})是一个用于构建动态分析工具的*nix仪器化框架,这意味着它可以在程序运行时执行分析。它附带了一系列工具,用于各种类型的调查和检查。其中一些工具包括:

\begin{itemize}
\item
Memcheck: 检测内存管理问题

\item
Cachegrind: 分析CPU缓存,识别缓存未命中和其他问题

\item
Callgrind: Cachegrind的扩展,提供关于调用图的额外信息

\item
Massif: 堆分析器,显示程序的不同部分随时间如何使用堆

\item
Helgrind: 用于数据竞争问题的线程调试器

\item
DRD: Helgrind的一个更轻量级、功能更有限的版本
\end{itemize}

在这个列表中的每个工具在需要时都非常有用。大多数系统包管理器都了解Valgrind,可以轻松地在您的操作系统上安装它。如果您使用的是Linux,它可能已经安装好了。此外,官方网站为那些喜欢自己构建的用户提供了源代码。

我们的讨论将主要关注Memcheck,这是Valgrind套件中最常用的工具(当开发者提到Valgrind时,他们通常指的是Valgrind的Memcheck)。我们将探讨如何使用CMake与之配合,这将使您在以后发现有必要时更容易采用套件中的其他工具。

\mySubsubsection{12.4.1.}{Memcheck}

Memcheck对于调试内存问题非常有价值,这个问题在C++中可能特别复杂。程序员对内存管理有广泛的控制权,可能会犯各种错误。这些错误可能包括读取未分配的或已释放的内存,多次释放内存,甚至写入错误的地址。这些错误很容易被忽视,甚至潜入到最简单的程序中。有时,只是一个被遗忘的变量初始化就足以遇到麻烦。

调用Memcheck的命令如下:

\begin{shell}
valgrind [valgrind-options] tested-binary [binary-options]
\end{shell}

Memcheck是Valgrind的默认工具,但您也可以像这样明确指定它:

\begin{shell}
valgrind --tool=memcheck tested-binary
\end{shell}

运行Memcheck可能会大大减慢您的程序速度;手册(请参阅“进一步阅读”中的链接)指出,使用它检测的程序可能会慢10-15倍。为了避免每次运行测试都等待Valgrind,我们将创建一个单独的目标,在需要测试代码时从命令行调用它。理想情况下,这应该在任何新代码合并到主代码库之前完成。您可以将此步骤包含在早期的Git钩子中,或者作为持续集成(CI)流程的一部分。

要在CMake中创建Valgrind的自定义目标,您可以在CMake生成阶段之后使用以下命令:

\begin{shell}
cmake --build <build-tree> -t valgrind
\end{shell}

以下是如何在CMake中添加此类目标:

\filename{ch12/03-valgrind/cmake/Valgrind.cmake}

\begin{cmake}
function(AddValgrind target)
    find_program(VALGRIND_PATH valgrind REQUIRED)
    add_custom_target(valgrind
        COMMAND ${VALGRIND_PATH} --leak-check=yes
        $<TARGET_FILE:${target}>
        WORKING_DIRECTORY ${CMAKE_BINARY_DIR}
    )
endfunction()
\end{cmake}

在这个例子中,我们定义了一个名为AddValgrind的CMake函数,它接受要测试的目标(我们可以在项目中重用它)。这里发生了两件主要的事情:

\begin{enumerate}
\item
CMake检查默认系统路径以查找valgrind可执行文件,并将其路径存储在VALGRIND\_PATH变量中。如果未找到二进制文件,则REQUIRED关键字将使配置出错停止。

\item
创建了一个名为valgrind的自定义目标。它对指定的二进制文件运行Memcheck,并始终检查内存泄漏的选项。
\end{enumerate}

可以在以下方式设置Valgrind选项:

\begin{itemize}
\item
在~/.valgrindrc文件中(在家目录中)

\item
通过\$VALGRIND\_OPTS环境变量

\item
在./.valgrindrc文件中(在工作目录中)
\end{itemize}

这些选项按该顺序检查。另外,请注意,只有当最后一个文件属于当前用户、是常规文件且未标记为全局可写时,才会考虑该文件。这是一种安全机制,因为给Valgrind的选项可能具有潜在的危害。

要使用AddValgrind函数,我们将其与unit\_tests目标一起提供,因为我们希望在一个精细控制的环境中(如单元测试)运行它:

\filename{ch12/03-valgrind/test/CMakeLists.txt (片段)}

\begin{cmake}
# ...
add_executable(unit_tests calc_test.cpp run_test.cpp)
# ...
include(Valgrind)
AddValgrind(unit_tests)
\end{cmake}

记住,使用Debug配置生成构建树可以让Valgrind利用调试信息,使其输出更加清晰。

让我们看看这在实践中是如何工作的:

\begin{shell}
# cmake -B <build tree> -S <source tree> -DCMAKE_BUILD_TYPE=Debug
# cmake --build <build-tree> -t valgrind
\end{shell}

这将配置项目,构建sut和unit\_tests目标,并启动Memcheck的执行,它将为我们提供一般信息:

\begin{shell}
[100%] Built target unit_tests
==954== Memcheck, a memory error detector
==954== Copyright (C) 2002-2017, and GNU GPL'd, by Julian Seward et al.
==954== Using Valgrind-3.18.1 and LibVEX; rerun with -h for copyright info
==954== Command: ./unit_tests
\end{shell}

前缀==954==包含进程ID,有助于区分Valgrind的评论和测试进程的输出。

接下来,像往常一样使用gtest运行测试:

\begin{shell}
[==========] Running 3 tests from 2 test suites.
[----------] Global test environment set-up.
...
[==========] 3 tests from 2 test suites ran. (42 ms total)
[ PASSED ] 3 tests.
\end{shell}

最后,呈现了一个摘要:

\begin{shell}
==954==
==954== HEAP SUMMARY:
==954==     in use at exit: 1 bytes in 1 blocks
==954==   total heap usage: 209 allocs, 208 frees, 115,555 bytes allocated
\end{shell}

哎呀!我们至少还在使用1字节。使用malloc()和new进行的分配没有与相应的free()和delete操作匹配。看起来我们的程序中有内存泄漏。Valgrind提供了更多细节来找到它:

\begin{shell}
==954== 1 bytes in 1 blocks are definitely lost in loss record 1 of 1
==954==   at 0x483BE63: operator new(unsigned long) (in /usr/lib/x86_64-
linux-gnu/valgrind/vgpreload_memcheck-amd64-linux.so)
==954==   by 0x114FC5: run() (run.cpp:6)
==954==   by 0x1142B9: RunTest_RunOutputsCorrectEquations_
Test::TestBody() (run_test.cpp:14)
\end{shell}

以0x<address>开头的行指示调用堆栈中的单个函数。我截断了输出(它包含了一些来自GTest的噪音),以专注于有趣的部分——最顶层的函数和源引用,run()(run.cpp:6):

最后,在底部找到了摘要:

\begin{shell}
==954== LEAK SUMMARY:
==954==    definitely lost: 1 bytes in 1 blocks
==954==    indirectly lost: 0 bytes in 0 blocks
==954==      possibly lost: 0 bytes in 0 blocks
==954==    still reachable: 0 bytes in 0 blocks
==954==         suppressed: 0 bytes in 0 blocks
==954==
==954== ERROR SUMMARY: 1 errors from 1 contexts (suppressed: 0 from 0)
\end{shell}

Valgrind在查找复杂问题方面非常出色。有时,它甚至可以更深入地挖掘,找到不容易归类的问题。这些将在“可能丢失”行中显示。

让我们看看Memcheck在这个案例中找到的问题是什么:

\filename{ch12/03-valgrind/src/run.cpp}

\begin{cpp}
#include <iostream>
#include "calc.h"
using namespace std;
int run() {
    auto c = new Calc();
    cout << "2 + 2 = " << c->Sum(2, 2) << endl;
    cout << "3 * 3 = " << c->Multiply(3, 3) << endl;
    return 0;
}
\end{cpp}

没错:突出显示的代码是有问题的。实际上,我们创建了一个在测试结束前没有被删除的对象。这正是为什么拥有广泛的测试覆盖非常重要的原因。

Valgrind是一个有用的工具,但在复杂的程序中,其输出可能会变得令人不知所措。有一种更有效管理这些信息的方法——那就是Memcheck-Cover项目。

\mySubsubsection{12.4.2.}{Memcheck-Cover}

像CLion这样的商业IDE可以直接解析Valgrind的输出,这使得通过图形界面导航变得更加容易,而无需在控制台中滚动。如果您的编辑器缺少这个功能,第三方报告生成器可以提供更清晰的视图。David Garcin开发的Memcheck-Cover通过创建一个HTML文件提供了更好的体验,如下面的图形所示:

\myGraphic{0.9}{content/chapter12/images/1.png}{图12.1:由Memcheck-Cover生成的报告}

这个整洁的小项目在GitHub上可用(\url{https://github.com/Farigh/memcheck-cover});它需要Valgrind和gawk(GNU AWK工具)。要使用它,我们将准备一个单独的CMake模块中的设置函数。它将包括两部分:

\begin{enumerate}
\item
获取和配置工具

\item
添加一个自定义目标来运行Valgrind并生成报告
\end{enumerate}

配置如下所示:

\filename{ch12/04-memcheck/cmake/Memcheck.cmake}

\begin{cmake}
function(AddMemcheck target)
    include(FetchContent)
    FetchContent_Declare(
        memcheck-cover
        GIT_REPOSITORY https://github.com/Farigh/memcheck-cover.git
        GIT_TAG release-1.2
    )
    FetchContent_MakeAvailable(memcheck-cover)
    set(MEMCHECK_PATH ${memcheck-cover_SOURCE_DIR}/bin)
\end{cmake}

在第一部分中,我们遵循与常规依赖项相同的做法:包括FetchContent模块,并使用FetchContent\_Declare指定项目的仓库和所需的Git标签。接下来,我们启动fetch过程并配置二进制路径,使用由FetchContent\_Populate设置的memcheckcover\_SOURCE\_DIR变量(由FetchContent\_MakeAvailable隐式调用)。

函数的第二部分是创建生成报告的目标。我们将称之为memcheck(这样它就不会与之前的valgrind目标重叠,如果我们想保留两个选项的某些原因):

\filename{ch12/04-memcheck/cmake/Memcheck.cmake (continued)}

\begin{cmake}
    add_custom_target(memcheck
        COMMAND ${MEMCHECK_PATH}/memcheck_runner.sh -o
            "${CMAKE_BINARY_DIR}/valgrind/report"
            -- $<TARGET_FILE:${target}>
        COMMAND ${MEMCHECK_PATH}/generate_html_report.sh
            -i "${CMAKE_BINARY_DIR}/valgrind"
            -o "${CMAKE_BINARY_DIR}/valgrind"
        WORKING_DIRECTORY ${CMAKE_BINARY_DIR}
    )
endfunction()
\end{cmake}

这发生在两个命令中:

\begin{enumerate}
\item
首先,我们将运行memcheck\_runner.sh包装脚本,它将执行Valgrind的Memcheck并收集输出到由-o参数提供的文件中。

\item
然后,我们将使用generate\_html\_report.sh解析输出并创建报告。此脚本需要由-i和-o参数提供的输入和输出目录。
\end{enumerate}

这两个步骤都应该在CMAKE\_BINARY\_DIR工作目录中执行,以便单元测试二进制文件可以通过相对路径访问文件,如果需要的话。

我们当然需要添加到我们的listfiles的最后一点是,当然,是对这个函数的调用:

\filename{ch12/04-memcheck/test/CMakeLists.txt (片段)}

\begin{cmake}
include(Memcheck)
AddMemcheck(unit_tests)
\end{cmake}

在生成带有Debug配置的构建系统后,我们可以使用以下命令构建目标:

\begin{shell}
# cmake -B <build tree> -S <source tree> -DCMAKE_BUILD_TYPE=Debug
# cmake --build <build-tree> -t memcheck
\end{shell}

然后,我们可以享受我们的格式化报告,生成为HTML页面。






“你将在阅读代码上花费的时间比编写代码的时间要多,应该优先优化代码的可读性。”这一原则在许多关于整洁代码的书籍中都有所体现。它得到了许多软件开发者的经验支持,这也是为什么像空格数量、换行,以及\#import语句的顺序这样的小细节都要标准化。这种标准化不仅仅是为了做到细致,而是为了节省时间。遵循本章中的实践,可以忘记手动格式化代码。构建时,代码会自动格式化,这是无论如何都要进行的测试代码的步骤。使用ClangFormat,可以确保格式化符合选择的标准。

除了简单的空格调整,代码还应该满足许多其他指南。这时clang-tidy就派上用场了,其有助于执行团队或组织约定的编码规范。我们深入讨论了这款静态检查器,还提到了其他选项,如Cpplint、Cppcheck、include-what-you-use和Link What You Use。由于静态链接器相对较快,可以几乎不投入成本地将其添加到构建中,这通常非常值得。

我们还检查了Valgrind工具,特别是Memcheck,用于识别内存管理问题,如错误的读写操作。这个工具在避免手动调试数小时,和防止生产环境中出现漏洞方面,具有无法估量的价值。我们介绍了一种方法,通过Memcheck-Cover这个HTML报告生成器,使Valgrind的输出更加用户友好。这在无法运行IDE的环境中使用特别有用,比如CI流程。

本章只是一个起点。还有许多其他免费和商业工具有助于提高代码质量。探索它们,找到最适合你的工具。下一章中,我们将深入探讨生成文档的内容。
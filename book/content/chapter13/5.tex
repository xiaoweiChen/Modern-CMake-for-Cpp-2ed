In this chapter, we delved into the practicalities of adding Doxygen, a powerful documentation generation tool, to your CMake project and enhancing its appeal. This task, though seemingly daunting, is quite manageable and significantly enhances the flow and clarity of information within your solution. As you’ll find, the time invested in adding and maintaining documentation is a worthwhile effort, especially when you or your teammates grapple with understanding complex relationships in the application. After exploring how to use CMake’s built-in Doxygen support to generate documentation in practice, we took a slight turn, to ensure not only the readability of the documentation but also its legibility.

Since dated design can be difficult on the eye, we explored alternative looks of the produced HTML. This was done using the Doxygen Awesome extension. To enable enhancements it comes with, we customized the standard header by adding the necessary javascript.

By generating documentation, you ensure its proximity to the actual code, making it easier to maintain written explanations in sync with the logic, especially if they’re both in the same file. Also, as a programmer, you’re likely juggling numerous tasks and details. Documentation acts as a memory aid, helping you retain and recall project intricacies. Keep in mind that even “the shortest pencil is longer than the longest memory.” Do yourself a favor—write long things down, and prosper.

Wrapping up, this chapter emphasizes the value of Doxygen in your project management toolkit, aiding both understanding and communication within your team.

In the next chapter, I’ll take you through automating packaging and the installation of projects with CMake, further enhancing your project management skills.
CMake提供了多种方式通过CMAKE\_变量、ENV变量和特殊命令来查询环境信息。例如,收集的信息可以用来支持跨平台的脚本。这些机制可以避免使用可能不易移植,或在不同环境中命名不同的特定平台的shell命令。

对于性能敏感型应用,了解构建平台的所有特性(例如,指令集、CPU核心数等)将非常有用。然后,将这些信息传递给编译后的二进制文件,以便进行调优。

现在,就来探索CMake可提供的本地信息。

\mySubsubsection{4.5.1.}{检测操作系统}

了解目标操作系统是什么很有用。即使是像文件系统这样的事物,在Windows和Unix之间也有很大差异,比如:大小写敏感性、文件路径结构、扩展名、权限等。一个系统上存在的命令在另一个系统上可能不可用,可能命名不同(例如,Unix中的ifconfig和Windows中的ipconfig)或者输出不同的内容。

如果需要用单个CMake脚本来支持多个目标操作系统,只需检查CMAKE\_SYSTEM\_NAME变量,以便采取相应地行动。

以下是一个简单的例子:

\begin{cmake}
if(CMAKE_SYSTEM_NAME STREQUAL "Linux")
    message(STATUS "Doing things the usual way")
elseif(CMAKE_SYSTEM_NAME STREQUAL "Darwin")
    message(STATUS "Thinking differently")
elseif(CMAKE_SYSTEM_NAME STREQUAL "Windows")
    message(STATUS "I'm supported here too.")
elseif(CMAKE_SYSTEM_NAME STREQUAL "AIX")
    message(STATUS "I buy mainframes.")
else()
    message(STATUS "This is ${CMAKE_SYSTEM_NAME} speaking.")
endif()
\end{cmake}

如果需要,有一个包含操作系统版本的变量:CMAKE\_SYSTEM\_VERSION。然而,我的建议是尽量使解决方案与系统无关,并使用CMake内置的跨平台功能。特别是对于文件系统的操作,应该使用附录中描述的file()命令。

\mySubsubsection{4.5.2.}{交叉编译——什么是主机和目标系统?}

交叉编译是指在一个机器上编译代码以在另一个目标平台上执行的过程。例如,使用适当的工具集,可以在Windows机器上运行CMake来编译Android应用程序。尽管交叉编译超出了本书的范围,但了解它如何影响CMake也很重要。

允许交叉编译的必要步骤是将CMAKE\_SYSTEM\_NAME和CMAKE\_SYSTEM\_VERSION变量设置为目标操作系统的适当值(CMake文档中将其称为目标系统)。执行构建的操作系统称为主机系统。

无论配置如何,主机系统的信息总是可以在带有HOST关键字的变量名称中访问: CMAKE\_HOST\_SYSTEM, CMAKE\_HOST\_SYSTEM\_NAME, CMAKE\_HOST\_SYSTEM\_PROCESSOR 和 CMAKE\_HOST\_SYSTEM\_VERSION。

还有一些带有HOST关键字的变量,其明确地引用主机系统。否则,所有变量都引用目标系统(通常是主机系统,除非在进行交叉编译)。

如果对交叉编译感兴趣,我建议参考CMake文档:\url{https://cmake.org/cmake/help/latest/manual/cmake-toolchains.7.html}。

\mySubsubsection{4.5.3.}{简写变量}

CMake将预定义一些变量,提供关于主机和目标系统的信息。如果使用了特定的系统,相应的变量将设置为非假值(即1或true):

\begin{itemize}
\item
ANDROID, APPLE, CYGWIN, UNIX, IOS, WIN32, WINCE, WINDOWS\_PHONE

\item
CMAKE\_HOST\_APPLE,CMAKE\_HOST\_SOLARIS,CMAKE\_HOST\_UNIX, CMAKE\_HOST\_WIN32
\end{itemize}

WIN32和CMAKE\_HOST\_WIN32变量对于32位和64位版本的Windows,以及MSYS都将为真(这个值因历史原因而保留)。此外,UNIX对于Linux、macOS和Cygwin也将为真。

\mySubsubsection{4.5.4.}{主机系统信息}

CMake可以提供更多的变量,为了节省时间,不会查询环境中很少需要的信息,比如处理器是否支持MMX,或者总物理内存是多少。但这些信息不可用——只需要使用以下命令明确地请求:

\begin{shell}
cmake_host_system_information(RESULT <VARIABLE> QUERY <KEY>...)
\end{shell}

需要提供一个目标变量和我们感兴趣的关键字列表。如果只提供一个关键字,变量将包含一个值;否则,它将是一个值列表。我们可以询问有关环境和操作系统的许多信息:

% Please add the following required packages to your document preamble:
% \usepackage{longtable}
% Note: It may be necessary to compile the document several times to get a multi-page table to line up properly
\begin{longtable}{|l|l|}
\hline
\textbf{关键字}                & \textbf{描述}                                      \\ \hline
\endfirsthead
%
\endhead
%
HOSTNAME                    & 主机名                                         \\ \hline
FQDN                        & 完全限定域名                      \\ \hline
TOTAL\_VIRTUAL\_MEMORY      & 以MiB为单位的虚拟内存总量                      \\ \hline
AVAILABLE\_VIRTUAL\_MEMORY  & 以MiB为单位的可用虚拟内存                  \\ \hline
TOTAL\_PHYSICAL\_MEMORY     & 以MiB为单位的总物理内存                     \\ \hline
AVAILABLE\_PHYSICAL\_MEMORY & 以MiB为单位的可用物理内存                 \\ \hline
OS\_NAME     & \begin{tabular}[c]{@{}l@{}}如果存在,则输出uname -s;\\无论是Windows、Linux,还是Darwin\end{tabular} \\ \hline
OS\_RELEASE                 & 操作系统子类型,如Windows Professional \\ \hline
OS\_VERSION                 & 操作系统构建ID                                  \\ \hline
OS\_PLATFORM & \begin{tabular}[c]{@{}l@{}}在Windows上和\$ENV\{PROCESSOR\_ARCHITECTURE\}\\的值一样. 在 Unix/macOS上和 uname -m 一样  \end{tabular}        \\ \hline
\end{longtable}

如果需要,甚至可以查询特定于处理器的信息:

% Please add the following required packages to your document preamble:
% \usepackage{longtable}
% Note: It may be necessary to compile the document several times to get a multi-page table to line up properly
\begin{longtable}{|l|l|}
\hline
\textbf{关键字}                & \textbf{描述}                                   \\ \hline
\endfirsthead
%
\endhead
%
NUMBER\_OF\_LOGICAL\_CORES  & 逻辑核数                       \\ \hline
NUMBER\_OF\_PHYSICAL\_CORES & 物理核数                      \\ \hline
HAS\_SERIAL\_NUMBER         & 如果处理器有序列号,则为1        \\ \hline
PROCESSOR\_SERIAL\_NUMBER   & 处理器序列号                       \\ \hline
PROCESSOR\_NAME             & 可读的处理器名称                 \\ \hline
PROCESSOR\_DESCRIPTION      & 可读的完整处理器描述     \\ \hline
IS\_64BIT                   & 如果处理器是64位的为1                      \\ \hline
HAS\_FPU                    & 如果处理器有浮点单元为1     \\ \hline
HAS\_MMX                    & 如果处理器支持MMX指令为1      \\ \hline
HAS\_MMX\_PLUS              & 如果处理器支持Ext. MMX指令为1 \\ \hline
HAS\_SSE                    & 如果处理器支持SSE指令为1      \\ \hline
HAS\_SSE2                   & 如果处理器支持SSE2指令为1     \\ \hline
HAS\_SSE\_FP                & 如果处理器支持SSE FP指令为1   \\ \hline
HAS\_SSE\_MMX               & 如果处理器支持SSE MMX指令为1  \\ \hline
HAS\_AMD\_3DNOW             & 如果处理器支持3DNow指令为1    \\ \hline
HAS\_AMD\_3DNOW\_PLUS       & 如果处理器支持3DNow+指令为1   \\ \hline
HAS\_IA64                   & 如果IA64处理器模拟x86,则为1          \\ \hline
\end{longtable}

\mySubsubsection{4.5.5.}{平台是32位,还是64位架构?}

64位架构中,内存地址、处理器寄存器、处理器指令、地址总线和数据总线都是64位宽的。虽然这是一个简化的定义,但它给出了64位平台与32位平台不同的一个大致概念。

在C++中,不同的架构意味着一些基本数据类型(int和long)和指针有不同的位宽。CMake利用指针大小来收集有关目标机器的信息。这些信息可以通过CMAKE\_SIZEOF\_VOID\_P变量获得,对于64位,包含值为8(因为指针是8字节宽);对于32位,包含值为4(4字节):

\begin{cmake}
if(CMAKE_SIZEOF_VOID_P EQUAL 8)
    message(STATUS "Target is 64 bits")
endif()
\end{cmake}

\mySubsubsection{4.5.6.}{系统的字节序是什么?}

架构可以根据字内字节顺序,或处理器的自然数据单位,分为大端序或小端序。在大端序系统中,最高有效字节存储在最低的内存地址,而最低有效字节存储在最高的内存地址。相反,在小端序系统中,字节顺序反转,最低有效字节存储在最低的内存地址,而最高有效字节存储在最高的内存地址。

大多数情况下,字节序并不重要,但是当编写需要可移植性的位操作代码时,CMake将提供BIG\_ENDIAN或LITTLE\_ENDIAN值,这些值存储在CMAKE\_<LANG>\_BYTE\_ORDER变量中,其中<LANG>是C、CXX、OBJC或CUDA。

已经了解了如何查询环境,现在让我们将重点转移到项目的主要配置上。






















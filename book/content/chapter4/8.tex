本章中,包含了为构建健壮且面向未来的项目打下坚实基础的重要概念。讨论了设置最低CMake版本和配置项目,如名称、语言和元数据字段。这些可使项目能够有效地扩展。

探讨了项目分区,比较了基本的include()与add\_subdirectory的使用,后者提供了诸如范围变量管理、简化路径和提高模块化等好处。创建嵌套项目并分别构建它们的能力,在逐渐将代码分解为更独立的单元时。证明了其价值。在理解了分区机制后,又深入研究了如何创建透明、健壮且可扩展的项目结构。检查了CMake遍历列表文件和配置步骤的正确顺序,并研究了如何限定目标机器和宿主机器的环境,它们之间的区别是什么,以及可以通过不同的查询获得关于平台和系统的哪些信息。我们还了解了配置工具链,包括指定所需的C++版本,处理特定供应商的编译器扩展,以及启用重要的优化。最后,了解了如何测试编译器所需的功能,并执行示例文件以测试编译支持。

目前为止,了解的技术对于项目至关重要,但它们不足以使项目真正有用。为了增加项目的实用性,需要理解目标的概念。我们之前简要地触及了这个话题,而现在,已经对相关基础知识有了扎实的理解,就可以对其进行全面地探讨。

下一章将介绍的目标,将在进一步增强我们项目的功能性和有效性方面发挥关键作用。
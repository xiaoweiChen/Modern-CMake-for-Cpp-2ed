在本章中,我们涵盖了为构建健壮且面向未来的项目打下坚实基础的重要概念。我们讨论了设置最低CMake版本和配置项目的基本方面,如名称、语言和元数据字段。建立这些基础使我们的项目能够有效地扩展。

我们探讨了项目分区,比较了基本的include()与add\_subdirectory的使用,后者提供了诸如范围变量管理、简化路径和提高模块化等好处。创建嵌套项目并分别构建它们的能力在逐渐将代码分解为更独立的单元时证明了其有用性。在理解了可用的分区机制后,我们深入研究了如何创建透明、健壮且可扩展的项目结构。我们检查了CMake遍历列表文件和配置步骤的正确顺序。接下来,我们研究了如何限定目标机器和宿主机器的环境,它们之间的区别是什么,以及可以通过不同的查询获得关于平台和系统的哪些信息。我们还涵盖了配置工具链,包括指定所需的C++版本,处理特定供应商的编译器扩展,以及启用重要的优化。我们学习了如何测试编译器所需的功能并执行示例文件以测试编译支持。

尽管到目前为止覆盖的技术方面对于项目至关重要,但它们不足以使项目真正有用。为了增加项目的实用性,我们需要理解目标的概念。 我们之前简要地触及了这个话题,但现在我们终于准备好全面地探讨它,因为我们已经对相关基础知识有了扎实的理解。下一章将介绍的目标将在进一步增强我们项目的功能性和有效性方面发挥关键作用。
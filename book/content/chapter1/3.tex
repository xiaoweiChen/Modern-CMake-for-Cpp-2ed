
CMake 是一个跨平台的、开源的软件,用 C++ 编写。我们可以自己编译它;然而,最好不要这样做。可以从官方网站 \url{https://cmake.org/download/} 下载预编译的二进制文件。

基于 Unix 的系统可以直接从命令行提供准备好的安装包。

\begin{myNotic}{Note}
CMake 并不包含编译器。如果系统还没有安装编译器,在使用 CMake 之前需要进行安装。确保将它们的执行文件路径添加到 PATH 环境变量中,这样 CMake 才能找到。

为了避免在学习本书时遇到工具和依赖问题,我建议通过第一种安装方法进行实践:Docker。在现实世界的场景中,你当然会想要使用本地的版本,除非你最初就在虚拟化的环境中。
\end{myNotic}

来看看 CMake 可以使用的环境。

\mySubsubsection{1.3.1}{Docker}

Docker (\url{https://www.docker.com/}) 是一个跨平台的工具,提供操作系统级别的虚拟化,允许应用程序以定义良好的包形式(称为容器)进行传输。这些是自给自足的捆绑包,包含运行所需的所有库、依赖项和工具。Docker 在轻量级环境中执行其容器,彼此隔离。

这个概念使得分享完成特定过程所需的所有工具链变得极为方便,无需担心微小的环境差异。

Docker 平台有一个公共的容器镜像仓库,\url{https://registry.hub.docker.com/},提供数百万个现成的镜像。

方便起见,我发布了两个 Docker 镜像:

\begin{itemize}
\item
swidzinski/cmake2:base: 基于 Ubuntu 的镜像,包含构建时所需的精心挑选的工具和依赖项

\item
swidzinski/cmake2:examples: 基于上述工具链的镜像,包含本书中的所有项目和示例
\end{itemize}

第一个选项,是为那些想有一个干净的镜像来构建项目的读者准备,第二个选项是为我们在章节中进行示例实践而准备。

可以按照官方文档中的说明安装 Docker(请参考 docs.docker.com/get-docker)。然后,在终端中执行以下命令来下载镜像,并启动容器:

\begin{shell}
$ docker pull swidzinski/cmake2:examples
$ docker run -it swidzinski/cmake2:examples
root@b55e271a85b2:root@b55e271a85b2:#
\end{shell}

请注意,示例位于以下格式的目录中

\begin{shell}
devuser/examples/examples/ch<N>/<M>-<title>
\end{shell}

<N>和<M>分别是零填充的章节和示例编号(例如 01, 08, 和 12)

\mySubsubsection{1.3.2}{Windows}

Windows 上安装CMake非常简单——只需从官方网站下载适用于 32 位或 64 位的版本即可。还可以选择适用于 Windows Installer 的便携式 ZIP 或 MSI 包装包,它会将 CMake 的 bin 目录添加到 PATH 环境变量中(图 1.2),这样一来就可以在任何目录中使用它,而不会出现此类错误:

\begin{shell}
'cmake' 不是内部或外部命令,也不是可运行的程序或批处理文件。
\end{shell}

如果选择 ZIP 包,需要手动安装。MSI 安装程序带有方便的 GUI:

\myGraphic{0.8}{content/chapter1/images/2.png}{图 1.2:安装向导可以设置环境变量 PATH}

这是一个开源软件,所以可以自己构建 CMake。然而,在 Windows 上,首先需要在系统上获取 CMake 的二进制文件,从而生成新版本的CMake。

Windows 平台与其他平台没有什么不同,也需要一个构建工具来完成由 CMake 开始构建过程。这里的一个流行选择是 Visual Studio IDE,它附带了 C++ 编译器。社区版可以从 Microsoft 的网站免费获得:\url{https://visualstudio.microsoft.com/downloads/}.

\mySubsubsection{1.3.3}{Linux}

Linux 上安装 CMake 的过程与安装其他软件包相同:命令行调用包管理器。包仓库通常会更新到 CMake 的最新版本,但通常不是最新版本。如果满意于此,并且使用像 Debian 或 Ubuntu 这样的发行版,最简单的方法就是直接安装适当的包:

\begin{shell}
$ sudo apt-get install cmake
\end{shell}

对于 Red Hat 发行版,使用以下命令:

\begin{shell}
$ yum install cmake
\end{shell}

\begin{myTip}{Tip}
当安装包时,包管理器将获取为操作系统配置的仓库中可用的最新版本。在许多情况下,包仓库不提供最新版本,而是提供经过时间考验的稳定版本,以确保可靠的工作。根据需求选择,但旧版本可能不包含本书中描述的某些功能。
\end{myTip}

要获取最新版本,请参考官方 CMake 网站的下载部分。如果知道当前版本号,可以使用以下命令。

Linux x86\_64 的命令:

\begin{shell}
$ VER=3.26.0 && wget https://github.com/Kitware/CMake/releases/download/
v$VER/cmake-$VER-linux-x86_64.sh && chmod +x cmake-$VER-linux-x86_64.sh &&
./cmake-$VER-linux-x86_64.sh
\end{shell}

Linux AArch64 的命令:

\begin{shell}
$ VER=3.26.0 && wget https://github.com/Kitware/CMake/releases/download/
v$VER/cmake-$VER-Linux-aarch64.sh && chmod +x cmake-$VER-Linux-aarch64.sh
&& ./cmake-$VER-Linux-aarch64.sh
\end{shell}

或者,查看从源代码构建部分,了解如何相应的平台上自行编译 CMake。

\mySubsubsection{1.3.4}{macOS}

这个平台也得到了 CMake 开发者的强烈支持。最流行的安装方法是通过 MacPorts:

\begin{shell}
$ sudo port install cmake
\end{shell}

撰写本文时,MacPorts 中可用的最新版本是 3.24.4。要获取最新版本,请安装 cmake-devel 包:

\begin{shell}
$ sudo port install cmake-devel
\end{shell}

或者,使用 Homebrew 包管理器:

\begin{shell}
$ brew install cmake
\end{shell}

macOS 包管理器将涵盖所有必要步骤,除非从源代码构建,否则无法获得最新版本。

\mySubsubsection{1.3.5}{从源代码构建}

如果使用其他平台,或者只是想体验尚未发布(或被你最喜欢的包仓库采用)的最新构建,请从官方网站下载源代码并自行编译:

\begin{shell}
$ wget https://github.com/Kitware/CMake/releases/
download/v3.26.0/cmake-3.26.0.tar.gz
$ tar xzf cmake-3.26.0.tar.gz
$ cd cmake-3.26.0
$ ./bootstrap
$ make
$ make install
\end{shell}

从源代码构建相对较慢且需要更多步骤,没有其他方法可以自由选择 CMake 的版本。这对于系统包仓库中的包陈旧时特别有用:系统版本越旧,获得的更新就越少。

现在我们已经安装了 CMake,来看看怎么使用它吧!






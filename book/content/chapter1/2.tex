C++源代码的编译看起很直接,我们从经典的Hello World示例开始。

以下代码位于ch01/01-hello/hello.cpp:

\begin{cpp}
#include <iostream>
int main() {
    std::cout << "Hello World!" << std::endl;
    return 0;
}
\end{cpp}

要生成一个可执行文件,需要一个C++编译器。CMake不包含编译器,所以需要选择并安装一个。主流的选择包括:

\begin{itemize}
\item
Microsoft Visual C++编译器

\item
GNU编译器集合

\item
Clang/LLVM
\end{itemize}

大多数读者都熟悉编译器,它是学习C++时不可或缺的,所以我们不会深入讨论选择和安装。本书中的示例将使用GNU GCC,它是一个成熟的开源软件编译器,可以在许多平台上免费获得。

假设已经安装了编译器,运行它对于大多数系统来说都类似:

\begin{shell}
$ g++ hello.cpp -o hello
\end{shell}

我们的代码正确,编译器会默默地产生一个可执行的二进制文件。现在,来运行它:

\begin{shell}
$ ./hello
Hello World!
\end{shell}

运行命令来构建你的程序很简单,但随着我们的项目增长,将所有内容放在一个文件中是不可能的。干净的代码实践建议,源代码文件应该保持小而有序。手动编译每个文件可能会变得脆弱。必须有更好的方法!

\mySubsubsection{1.2.1}{什么是CMake?}

我们通过编写一个脚本来自动化构建,该脚本遍历我们的项目树并编译所有内容。为了避免不必要的编译,脚本将检测源代码自上次运行脚本以来是否已修改。现在,希望有一种方便的方式来管理,传递给编译器的每个文件的参数——最好是基于可配置的标准。此外,脚本应该知道如何将所有编译文件,链接成一个单一的二进制文件。甚至更好地构建整个解决方案,这些解决方案可以作为更大的项目中的模块重复使用和集成。

软件构建是一个多样化的过程:

\begin{itemize}
\item
编译可执行文件和库

\item
管理依赖关系

\item
测试

\item
安装

\item
打包

\item
生成文档

\item
再测试
\end{itemize}

创建一个模块化和C++构建工具,以适应各种目的,需要很长时间。Kitware的Bill Hoffman在20多年前实现了CMake的第一个版本,它非常成功。今天,它具有许多功能和广泛的社区支持。CMake正在积极开发中,并已成为C和C++程序员行业标准。

自动化构建代码的问题比CMake还要古老,因此也有很多选择:GNU Make、Autotools、SCons、Ninja、Premake等。但CMake为什么能脱颖而出?

有几方面我觉得很重要:

\begin{itemize}
\item
专注于支持现代编译器和工具链。

\item
跨平台——支持为Windows、Linux、macOS和Cygwin构建。

\item
为流行的IDE生成项目文件:Microsoft Visual Studio、Xcode和Eclipse CDT。此外,它是其他项目的项目模型,如CLion。

\item
在抽象层次上操作——允许将文件分组为可重用的目标和项目。

\item
有大量使用CMake构建的项目,提供了一种简单的方式将它们集成到自己的项目中。

\item
将测试、打包和安装视为构建过程的固有部分。

\item
旧的、不常用的功能会废弃,以保持精简。
\end{itemize}

CMake提供了一致、流畅的体验。无论是使用IDE还是直接从命令行构建软件,最重要的是也照顾到构建后的阶段。

CI/CD流水可以使用相同的CMake配置和构建项目,即使所有先前环境都不同。

\mySubsubsection{1.2.2}{工作原理}

你可能认为CMake是一个工具,在一端读取源代码,在另一端生成二进制文件——尽管这在原则上是对的,但这并不是全部。

CMake不能独立构建——它依赖于系统中的其他工具来执行实际的编译、链接等任务。你可以把它看作是构建过程的指挥家:它知道需要完成哪些步骤,最终目标是什么,以及如何找到合适的工人和材料。

这个过程有三个阶段:

\begin{itemize}
\item
配置

\item
生成

\item
构建
\end{itemize}

让我们详细探讨它们。

\mySamllsection{配置阶段}

这个阶段是关于读取存储在目录中的项目信息,称为源码树,并为生成阶段准备一个输出目录或构建树。

CMake首先检查项目是否以前配置过,并从CMakeCache.txt文件中读取缓存的配置变量。首次运行时,它会创建一个空的构建树,并收集它正在工作的环境的所有详细信息:例如,架构是什么,可用的编译器是什么,以及是否安装了链接器和打包器。此外,还会检查一个简单的测试程序是否可以正确编译。

接下来,CMakeLists.txt项目配置文件被解析并执行(CMake项目使用CMake的编码语言进行配置)。这个文件是CMake项目的最小配置(源文件可以在稍后添加),告诉CMake关于项目结构、其目标和依赖项(库和其他CMake包)。

这个过程中,CMake将收集的信息存储在构建树中,例如系统详情、项目配置、日志和临时文件,这些信息会用于下一阶段。具体来说,CMakeCache.txt文件用于存储更稳定的信息(例如编译器和工具的路径),这在整个构建序列再次执行时可以节省时间。

\mySamllsection{生成阶段}

阅读项目配置后,CMake将为确切的环境生成一套构建系统。构建系统只是为其他构建工具(例如,GNU Make的Makefiles或Ninja,以及Visual Studio的IDE项目文件)定制的配置文件。在这个阶段,CMake仍然可以通过计算生成器表达式对构建配置进行一些调整。

\begin{myTip}{Tip}
生成阶段在配置阶段之后自动执行。因此,本书和其他资料有时会交替使用这两个阶段,当提到“配置”或“生成”构建系统时。要明确只运行配置阶段,可以使用cmake-gui工具。
\end{myTip}

\mySamllsection{构建阶段}

为了生成项目中指定的最终产品(如可执行文件和库),CMake必须运行适当的构建工具。这可以通过直接调用、通过IDE或使用适当的CMake命令来执行。反过来,这些构建工具将执行一系列步骤,使用编译器、链接器、静态和动态分析工具、测试框架、报告工具,以及其他工具来生成目标产品。

这种解决方案的美丽之处在于,能够根据需要为每个平台生成构建系统(使用相同的项目文件):

\myGraphic{0.9}{content/chapter1/images/1.png}{图1.1:CMake的各个阶段}

还记得之前在“理解基础知识”部分中的hello.cpp应用程序吗?使用CMake构建它真的非常简单。只需要在源文件所在的同一目录下创建以下CMakeLists.txt文件。

\filename{ch01/01-hello/CMakeLists.txt}

\begin{cmake}
cmake_minimum_required(VERSION 3.26)
project(Hello)
add_executable(Hello hello.cpp)
\end{cmake}

创建此文件后,在该目录下执行以下命令:

\begin{shell}
cmake -B <build tree>
cmake --build <build tree>
\end{shell}

请注意,<build tree>是一个占位符,应该用一个临时目录的路径替换,该目录将包含生成的文件。

以下是在Ubuntu系统上运行的Docker(Docker是一个可以在其他系统上运行的虚拟机)的输出。

第一个命令生成构建系统:

\begin{shell}
~/examples/ch01/01-hello# cmake -B ~/build_tree
-- The C compiler identification is GNU 11.3.0
-- The CXX compiler identification is GNU 11.3.0
-- Detecting C compiler ABI info
-- Detecting C compiler ABI info - done
-- Check for working C compiler: /usr/bin/cc - skipped
-- Detecting C compile features
-- Detecting C compile features - done
-- Detecting CXX compiler ABI info
-- Detecting CXX compiler ABI info - done
-- Check for working CXX compiler: /usr/bin/c++ - skipped
-- Detecting CXX compile features
-- Detecting CXX compile features - done
-- Configuring done (1.0s)
-- Generating done (0.1s)
-- Build files have been written to: /root/build_tree
\end{shell}

第二个命令构建项目:

\begin{shell}
~/examples/ch01/01-hello# cmake --build ~/build_tree
Scanning dependencies of target Hello
[ 50%] Building CXX object CMakeFiles/Hello.dir/hello.cpp.o
[100%] Linking CXX executable Hello
[100%] Built target Hello
\end{shell}

最后,运行编译后的程序:

\begin{shell}
~/examples/ch01/01-hello# ~/build_tree/Hello
Hello World!
\end{shell}

这里,在构建树目录中生成了一个构建系统。接着,执行了构建阶段,并产生了一个可以运行的二进制文件。

现在知道结果是什么样子了,我敢肯定你会有很多问题:这个过程的先决条件是什么?这些命令是什么意思?为什么需要两个?如何编写自己的项目文件?别担心——这些问题将在接下来的部分中得到解答。

\begin{myNotic}{Note}
本书将提供与当前版本的CMake(撰写时为3.26)相关的最重要信息。为了提供最佳建议,我明确避免了废弃和不推荐使用的功能,并强烈建议使用CMake 3.15或更新的版本(现代CMake)。如果需要更多信息,可以在网上找到最新的完整文档,网址为 \url{https://cmake.org/cmake/help/}。
\end{myNotic}

















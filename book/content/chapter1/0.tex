
软件开发的神奇之处在于,不仅是在创造能够运行的机制,还有创造解决方案的思想。

为了将想法变为现实,我们按照以下循环工作:设计、编码和测试。我们创造变化,并用编译器能理解的语言表达,继续检查其是否如期工作。要从源码中创建正确、高质量的软件,需要小心地重复执行容易出错的任务:调用正确的命令、检查语法、链接二进制文件、运行测试、报告问题等。

记住每个步骤需要付出巨大的努力。相反,我们希望更专注于编码,将其他事情委托给工具。理想情况下,这个过程会在更改代码后,通过一个按钮启动。这个过程应该是智能的、快速的、可扩展的,并且在不同操作系统和环境中的工作方式相同,应该得到多个集成开发环境(IDE)的支持。并且,可以将其简化为持续集成(CI)流水线,每次向仓库提交更改时,都会构建和测试软件。

CMake是满足许多此类需求的答案,但要正确配置和使用也要花一些心思。CMake并不是复杂性的来源,复杂性来自于要处理的东西。别担心,我们将系统地学习整个过程,你会了解到软件构建是多么“简单”。

我知道你急于开始编写自己的CMake项目,这正是本书大部分内容中做的事情。但是,由于你将主要为了用户(包括你自己)创建项目,因此了解一下他们的视角对你来说很重要。

我们从这一点开始,逐步成为一个CMake高级用户。我们将介绍一些基础知识:这个工具是什么,工作原理,以及如何安装。然后,将深入探讨命令行和操作模式。最后,将总结项目文件的不同用途,并解释如何在完全不创建项目的情况下使用CMake。

本章中,将包括以下主题:

\begin{itemize}
\item
基础知识

\item
安装CMake

\item
掌握命令行

\item
项目文件

\item
脚本和Find-模块
\end{itemize}

























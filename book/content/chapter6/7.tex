

This chapter was all about unpacking the ins and outs of generator expressions, or “genexes.” We started with the basics of forming and expanding generator expressions and looked at their nesting mechanism. We dove into the power of conditional expansion, which taps into Boolean logic, comparison operations, and queries. This aspect of generator expressions shines when adapting our build process based on factors like user-chosen build configuration, platform, and the current toolchain.

We have also covered the basic but essential transformations of strings, lists, and paths. A major highlight was using genexes to query the information gathered at the later build stages and present it when the context matches the requirements. We also now know how to check our compiler’s ID, version, and capabilities. We explored querying the build target properties and extracting the related information using generator expressions. The chapter is wrapped up with practical examples and guidance on viewing the output where possible. With this, you’re now ready to use the generator expressions in your projects.

In the next chapter, we’ll learn how to compile programs with CMake. Specifically, we’ll talk about how to configure and optimize this process.
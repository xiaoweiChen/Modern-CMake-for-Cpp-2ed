

本章全部内容都是关于剖析生成器表达式(或称“genexes”)的细节。我们从生成器表达式的形成和扩展的基础知识开始,并查了解了它的嵌套机制。深入探讨了条件扩展的强大功能,其利用了布尔逻辑、比较操作和查询。当根据用户选择的构建配置、平台和当前工具链等因素调整构建过程时,生成器表达式的这一方面尤为突出。

我们还了解了字符串、列表和路径的基本但重要的转换。一个重点亮点是使用查询在后期构建阶段收集的信息,并在上下文符合要求时呈现。现在也知道如何检查编译器的ID、版本和功能。探索了使用生成器表达式查询构建目标属性,并提取了相关信息。并以实用的示例和可能的输出查看指南作为结尾。有了这些,就可以在项目中使用生成器表达式了。

下一章中,将学习如何使用CMake编译程序。具体来说,将讨论如何配置和优化这个过程。
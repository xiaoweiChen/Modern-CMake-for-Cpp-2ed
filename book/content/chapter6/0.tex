许多CMake用户在他们的私人探索中并未遇到生成器表达式,因为这些概念相当高级。然而,对于准备进入普遍可用阶段或首次向更广泛受众发布的项目来说,它们在导出、安装和打包方面起着重要的作用。如果您只是想快速学习CMake的基础知识并专注于C++方面,可以暂时跳过本章,以后再回来阅读。另一方面,我们现在讨论生成器表达式,因为接下来的章节在解释CMake更深入的内容时将引用这些知识。

我们将从介绍生成器表达式的主题开始:它们是什么,它们有什么用途,以及它们是如何形成和扩展的。接下来将简要介绍嵌套机制,并对条件扩展进行更详细的描述,这允许使用布尔逻辑、比较操作和查询。当然,我们将深入探讨可用的表达式的大量内容。

但首先,我们将研究字符串、列表和路径的转换,因为在专注于主要内容之前,了解基础知识是很有用的。最终,生成器表达式在实际应用中被用来获取在构建后期阶段可用的信息,并在适当的上下文中呈现。确定这个上下文是它们价值的一个巨大部分。我们将发现如何根据用户选择的构建配置、当前平台和当前工具链来参数化我们的构建过程。也就是说,我们要确定正在使用的编译器、其版本以及它具有哪些功能,不仅如此:我们还将找出如何查询构建目标及其相关信息的属性。

为了确保我们能够充分理解生成器表达式的价值,我在本章的最后部分包含了一些有趣的使用示例。哦,还有一个关于如何查看生成器表达式输出的快速解释,因为这有点棘手。不过别担心,生成器表达式并不像它们看起来那么复杂,您很快就会使用它们。

在本章中,我们将包含以下内容:

\begin{itemize}
\item
生成器表达式是什么?

\item
学习通用表达式语法的基本规则

\item
条件展开

\item
查询和转换

\item
尝试示例
\end{itemize}


































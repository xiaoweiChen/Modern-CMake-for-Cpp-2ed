许多CMake用户在探索中并未遇到生成器表达式,这个概念相当高级。然而,对于准备进入普遍可用阶段,或首次向更广泛受众发布的项目来说,在导出、安装和打包方面起着重要的作用。如果只是想快速学习CMake的基础知识并专注于C++方面,可以暂时跳过本章,以后再回来阅读。另一方面,我们现在讨论生成器表达式,因为接下来的章节在解释CMake更深入的内容时,会引用这些知识。

我们将从介绍生成器表达式的主题开始:它们是什么,有什么用途,以及如何形成和扩展的。接下来将简要介绍嵌套机制,并对条件扩展进行更详细的描述,可以使用布尔逻辑、比较操作和查询。当然,会将深入探讨表达式的内容。

首先,研究字符串、列表和路径的转换,专注于主要内容之前,了解基础知识很有必要。最终,生成器表达式在实际应用中用来获取在构建后期阶段可用的信息,并在适当的上下文中呈现。确定这个上下文有很大的价值,将了解如何根据用户选择的构建配置、当前平台和当前工具链,来参数化构建过程。也就是说,要确定正在使用的编译器、其版本,以及具有的功能。不仅如此,还将找出如何查询构建目标及其相关信息的属性。

为了确保能够充分理解生成器表达式,我在本章的最后部分包含了一些有趣的使用示例。还有一个关于如何查看生成器表达式输出的快速解释,这有点棘手。不过别担心,生成器表达式并不像看起来那么复杂。

本章中,将包含以下内容:

\begin{itemize}
\item
生成器表达式是什么?

\item
通用表达式语法的基本规则

\item
条件扩展

\item
查询和转换

\item
尝试示例
\end{itemize}


































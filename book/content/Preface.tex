


创建顶级软件并非易事。在线研究这个主题的开发者常常难以确定哪些建议是当前的,哪些方法已经被更新、更好的实践所取代。此外,大多数资源以混乱的方式解释过程,缺乏适当的背景、上下文和结构。

《Modern CMake for C++》提供了一个端到端的指南,通过全面处理C++解决方案的构建,提供了更简单的体验。它不仅教你如何在项目中使用CMake,还强调了如何使项目保持可维护性、优雅和简洁。该指南引导你完成许多项目中常见的复杂任务的自动化,包括构建、测试和打包。

本书指导你如何组织源目录、构建目标和创建包。随着你的进步,你将学习编译和链接可执行文件和库,详细理解这些过程,并优化每个步骤以获得最佳结果。此外,你将发现如何将外部依赖项(如第三方库、测试框架、程序分析工具和文档生成器)整合到你的项目中。最后,你将学习如何导出、安装和打包你的解决方案,以供内部和外部使用。

完成这本书后,你将能够在专业水平上自信地使用CMake。

\myChapterNoFile{}{适读人群}

在你学会了C++之后,你很快就会发现,仅仅掌握语言本身并不足以让你准备好以最高标准交付项目。这本书填补了这一空白:它面向任何渴望成为更好的软件开发者甚至专业构建工程师的人!如果你想要从零开始学习现代CMake或提升和刷新你当前的CMake技能,就读这本书吧。它将帮助你了解如何制作顶级的C++项目,并从其他构建环境中过渡。

\myChapterNoFile{}{关于本书}

第1章,\textit{CMake入门的第一步},涵盖了CMake的安装、命令行界面的使用,并介绍了CMake项目所需的基本构建块。

第2章,\textit{CMake语言},涵盖了CMake语言的基本概念,包括命令调用、参数、变量、控制结构和注释。

第3章,\textit{在主流IDE中使用CMake},强调了集成开发环境(IDEs)的重要性,指导你选择IDE,并为Clion、Visual Studio Code和Visual Studio IDE提供设置说明。

第4章,设置你的第一个CMake项目,将教你如何在顶级文件中配置基本的CMake项目,结构化文件树,并准备必要的工具链进行开发。

第5章,\textit{了解目标},探讨了逻辑构建目标的概念,理解它们的属性和不同类型,并学习如何为CMake项目定义自定义命令。

第6章,\textit{生成器表达式},解释了生成器表达式的目的和语法,包括如何使用它们进行条件扩展、查询和转换。

第7章,\textit{编译C++源代码},深入探讨了编译过程,配置预处理器和优化器,并发现减少构建时间和提高调试的技术。

第8章,\textit{链接可执行文件和库},理解链接机制,不同类型的库,单一定义规则,链接顺序,以及如何准备你的项目进行测试。

第9章,\textit{CMake中管理依赖项},将教你如何管理第三方库,为那些缺乏CMake支持的库添加支持,并从互联网上获取外部依赖。

第10章,\textit{使用C++20模块},介绍了C++20模块,展示了如何在CMake中启用它们的支持,并相应地配置工具链。

第11章,\textit{测试框架},将帮助你理解自动化测试的重要性,利用CMake内置的测试支持,并使用流行的框架开始单元测试。

第12章,\textit{程序分析工具},将展示你如何自动格式化源代码并在构建时间和运行时检测软件错误。

第13章,\textit{生成文档},介绍了如何使用Doxygen从源代码自动创建文档,并添加样式以增强你的文档外观。

第14章,\textit{安装和打包},准备你的项目进行发布,无论是否安装,创建可重用包,并为打包指定单个组件。

第15章,\textit{创建专业项目},应用本书所学到的所有知识来开发一个全面、专业级别的项目。

第16章,\textit{编写CMake预设},将高级项目配置封装到使用CMake预设文件的工作流程中,使项目设置和管理更加高效。

附录 - 其他命令,作为与字符串、列表、文件和数学运算相关的各种CMake命令的参考。

\myChapterNoFile{}{利用本书}

本书假设你对C++和类Unix系统有基本的熟悉。尽管Unix知识不是严格的要求,但它将有助于你完全理解本书中给出的示例。

本书的目标是CMake 3.26,但这里描述的大多数技术应该适用于CMake 3.15(之后添加的功能通常会突出显示)。有些章节已经更新到CMake 3.28,以涵盖最新功能。

运行示例的环境准备在第1-3章中介绍,但如果你熟悉这个工具,我们特别推荐使用本书提供的Docker镜像。

\myChapterNoFile{}{下载示例}

本书的代码包托管在GitHub上,地址为\url{https://github.com/PacktPublishing/Modern-CMake-for-Cpp-2E}。我们还有其他丰富的书籍和视频代码包,也可在我们的GitHub仓库中找到,地址为:\url{https://github.com/PacktPublishing/}。去看看吧!

\myChapterNoFile{}{下载彩图}

我们还提供了一个PDF文件,其中包含了本书中使用的屏幕截图/图表的彩色图片。你可以在这里下载: \url{https://packt.link/gbp/9781805121800}

\myChapterNoFile{}{内容约定}

本书中使用了一些文本约定。

CodeInText: 表示文本中的代码单词、数据库表名、文件夹名、文件名、文件扩展名、路径名、虚拟URL、用户输入和Twitter句柄。例如:“将下载的WebStorm-10*.dmg磁盘映像文件作为系统中的另一个磁盘挂载。”

代码块如下设置:

\begin{cmake}
cmake_minimum_required(VERSION 3.26)
project(Hello)
add_executable(Hello hello.cpp)
\end{cmake}

当我们想要吸引你注意代码块中的特定部分时,相关的行或项目将以粗体显示:

\begin{cmake}
cmake_minimum_required(VERSION 3.26)
project(Hello)
add_executable(Hello hello.cpp)
add_subdirectory(api) # bold
\end{cmake}

命令行输入或输出如下:

\begin{shell}
cmake --build <dir> --parallel [<number-of-jobs>]
cmake --build <dir> -j [<number-of-jobs>]
\end{shell}

粗体(Bold):表示新术语、重要单词或你在屏幕上看到的单词。例如:“从管理面板中选择系统信息。

\begin{myNotic}{Note}
警告或重要注释
\end{myNotic}


\begin{myTip}{Tip}
提示和技巧
\end{myTip}







本章将构建的软件并不复杂——将创建一个简单的计算器,可以实现两个数字的相加(图15.1)。这是一个控制台应用程序,具有文本用户界面,利用第三方库和独立的计算库,这些库可以用于其他项目。尽管这个项目可能没有重要的实际应用,但其简单性非常适合演示本书讨论的各种技术应用。

\myGraphic{0.4}{content/chapter15/images/1.png}{图15.1:项目在支持鼠标的终端中执行的文本用户界面}

通常,项目要么生成面向用户的可执行文件,要么为开发者生成库。项目同时产生这两者的情况较少,尽管这种情况确实存在。例如,一些应用程序附带了独立的SDK或库以帮助开发插件。另一个例子是附带了使用示例的库。我们的项目属于后者,展示了库的功能。

我们将通过回顾章节列表,回忆每个章节的内容,并选择将用来构建应用程序的技术和工具来开始规划:

\begin{itemize}
\item
第1章,CMake入门:

本章提供了关于CMake的基本细节,包括安装和用于构建项目的命令行使用。还包含了关于项目文件的基本信息,如作用、典型的命名约定和特殊性。

\item
第2章,CMake语言:

我们介绍了编写正确的CMake列表文件和脚本所需的工具,包含了代码基础,如注释、命令调用和参数。我们解释了变量、列表和控制结构,引入了几个有用的命令。这个基础在我们的项目中至关重要。

\item
第3章,在IDE中使用CMake:

我们讨论了三个IDE——CLion、VS Code和Visual Studio IDE,介绍了它们的优点。在最终项目中,选择IDE(或不选择)由你决定。做出决定后,可以在这个项目中使用Dev容器开始,只需几个步骤就可以构建Docker镜像(或者直接从Docker Hub获取)。在容器中运行镜像可以确保开发环境与生产环境相匹配。

\item
第4章,设置CMake项目:

配置项目至关重要,其决定了将生效的CMake策略、命名、版本控制和编程语言。我们将使用本章来影响构建过程的基本行为。

我们还将遵循建立的项目分区和结构来确定目录和文件的布局,并利用系统发现变量以适应不同的构建环境。工具链配置是另一个关键方面,可以强制使用特定的C++版本和编译器支持的标准。按照章节的建议,我们将禁用源内构建以保持工作区清洁。

\item
第5章,使用目标:

了解到每个现代CMake项目都广泛使用目标,当然也会应用目标来定义一些库和可执行文件(用于测试和生产),这将使项目保持组织并确保我们遵循DRY(不要重复自己)的原则。对目标属性和传递使用要求(传播属性)的了解,将能够使配置接近目标定义。

\item
第6章,使用生成器表达式:

生成器表达式在项目中大量使用,力求使这些表达式尽可能简单。项目将包含自定义命令以生成Valgrind和覆盖率报告的文件。此外,还将使用目标钩子,特别是PRE\_BUILD,来清理覆盖率检测过程产生的.gcda文件。

\item
第7章,使用CMake编译C++源代码:

没有C++项目的编译是不可能的。基础知识相当简单,但CMake允许我们以许多方式调整这个过程:扩展目标源代码、配置优化器并提供调试信息。对于这个项目,默认的编译标志就可以了,也研究了一下预处理器:

\begin{itemize}
\item
我们将构建元数据(项目版本、构建时间和Git提交SHA)存储在编译后的可执行文件中并向用户展示。

\item
我们将启用预编译头文件。在如此小的项目中,这并不是真正必要的,但它将帮助我们练习这个概念。
\end{itemize}

不需要Unity构建。

\item
第8章,链接可执行文件和库:

我们将获得默认情况下对项目都有用的链接的一般信息。此外,由于这个项目包含一个库,将明确引用以下特定构建指令:

\begin{itemize}
\item
用于测试和开发的静态库

\item
用于发布的共享库
\end{itemize}

本章还概述了如何隔离main()函数以用于测试目的,我们将采用这种做法。

\item
第9章,管理依赖关系:

为了增强项目的吸引力,将引入一个外部依赖:一个基于文本的用户界面库。第9章探讨了管理依赖关系的各种方法。选择将很简单:FetchContent实用模块通常推荐且最方便。

\item
第10章,使用C++20模块:

尽管我们已经探讨了使用C++20模块,以及支持此功能的环境要求(CMake 3.28,最新编译器),但其广泛支持仍然不足。为了确保项目的可访问性,我们暂时不会引入模块。

\item
第11章:测试框架

实施适当的自动化测试,对于确保解决方案质量随时间保持一致至关重要。我们将集成CTest并组织项目以方便测试,并应用之前提到的main()函数分离方法。

本章将讨论两种测试框架:Catch2和GTest与GMock;我们将使用后者。为了获取覆盖率的详细信息,我们将使用LCOV生成HTML报告。

\item
第12章:程序分析工具

对于静态分析,可以从一系列工具中选择:Clang-Tidy、Cpplint、Cppcheck、include-what-you-use(IWYU)和link-what-you-use(LWYU)。我们将选择Cppcheck,因为Clang-Tidy与使用GCC构建的预编译头文件兼容性较差。

动态分析将使用Valgrind的Memcheck工具,并配合Memcheck-cover包装器来生成HTML报告。此外,在构建过程中,源码将自动通过ClangFormat进行格式化。

\item
第13章:文档生成

提供文档对于我们项目中的库来说是必不可少的。CMake支持使用Doxygen自动化生成文档。我们将采用这种方法,并在设计中加入doxygen-awesome-css主题以更新样式。

\item
第14章:安装与打包

最后,将配置解决方案的安装和打包,并准备文件形成包,包括目标定义。安装这些内容及构建目标产生的工件到合适的目录中,通过包含GNUInstallDirs模块实现。还将配置一些组件以模块化解决方案,并为CPack做好准备。
\end{itemize}

专业的项目通常会附带一些文本文件:README、LICENSE、INSTALL等。我们将在章节末尾简要介绍这些文件。

为了简化流程,不会实现自定义逻辑来检查所有必需的工具和依赖项是否可用。我们将依赖于CMake来显示其诊断信息并告诉用户缺少什么。如果项目获得了重要的关注,可能需要考虑添加这些机制以改善用户体验。

有了清晰的计划后,让我们讨论如何实际地构建项目结构,包括逻辑目标和目录结构。



































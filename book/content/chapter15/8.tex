本章中,基于目前为止所学的内容构建了一个专业项目。让我们快速回顾一下。

首先规划了项目的布局,并讨论了哪些文件将位于哪个目录中。基于以往的经验,以及想要实践更高级场景的愿望,界定了一个面向用户的主应用程序和一个其他开发者可能使用的库。这塑造了目录的结构,以及希望构建的 CMake 目标之间的关系。随后,配置了各个构建目标:提供了库的源代码,定义了它的目标,并设置了使用位置独立代码参数供使用。面向用户的应用程序也可以定义其可执行目标,提供了源代码,并配置了其依赖项:FTXUI 库。

有了待构建的制品之后,继续增强了项目的测试和质量保证功能。添加了覆盖率模块来生成覆盖率报告,使用 Memcheck 在运行时通过 Valgrind 验证解决方案,并执行了静态分析 CppCheck。

这样一个项目现在已经准备好安装,所以我们使用学到的技术为库和可执行文件创建了适当的安装条目,并为 CPack 准备了包配置。最后的任务是确保项目文档正确无误,因此设置了使用 Doxygen 自动生成文档,并编写了几份基本文档来处理软件分发中不太技术性的方面。

这就完成了项目配置,并且现在可以轻松地使用几个精确的 CMake 命令来构建和安装项目。但如果我们可以只用一个简单的命令来完成整个过程呢?让我们在最后一章:第 16 章中探索这个主题。
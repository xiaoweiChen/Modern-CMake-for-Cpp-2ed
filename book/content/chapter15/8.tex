在本章中,我们基于目前为止所学的内容构建了一个专业项目。让我们快速回顾一下。

我们首先规划了项目的布局,并讨论了哪些文件将位于哪个目录中。基于以往的经验以及想要实践更高级场景的愿望,我们界定了一个面向用户的主应用程序和一个其他开发者可能使用的库。这塑造了目录的结构以及我们希望构建的 CMake 目标之间的关系。随后,我们配置了各个构建目标:提供了库的源代码,定义了它的目标,并设置了使用位置独立代码参数供消费。面向用户的应用程序也有其可执行目标被定义,提供了源代码,并配置了其依赖项:FTXUI 库。

有了待构建的制品之后,我们继续增强了项目的测试和质量保证功能。我们添加了覆盖率模块来生成覆盖率报告,使用 Memcheck 在运行时通过 Valgrind 验证解决方案,并执行了静态分析 CppCheck。

这样一个项目现在已经准备好安装,所以我们使用学到的技术为库和可执行文件创建了适当的安装条目,并为 CPack 准备了包配置。最后的任务是确保项目文档正确无误,因此我们设置了使用 Doxygen 自动生成文档,并编写了几份基本文档来处理软件分发中不太技术性的方面。

这使我们完成了项目配置,并且我们现在可以轻松地使用几个精确的 CMake 命令来构建和安装项目。但如果我们可以只用一个简单的命令来完成整个过程呢?让我们在最后一章:第 16 章《编写 CMake 预设》中探索这一点。
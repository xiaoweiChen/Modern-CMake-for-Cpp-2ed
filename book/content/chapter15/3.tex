
为了构建任何项目,应该首先明确项目内部将创建哪些逻辑目标。这个案例中,将遵循下图所示的结构:

\myGraphic{0.8}{content/chapter15/images/2.png}{图 15.2: 逻辑目标的结构}

按照构建顺序来探索这个结构。首先,编译calc\_obj,一个对象库。然后,将注意力转向静态库和共享库。

\mySubsubsection{15.3.1.}{共享库与静态库}

第8章中,介绍了共享库和静态库。当多个程序使用相同的库时,共享库可以减少整体内存使用量,用户通常已经安装了流行的库或者知道如何快速安装它们。

更重要的是,共享库是独立的文件,必须放置在特定路径以便动态链接器能够找到它们。相比之下,静态库直接嵌入到可执行文件中,这使得使用更快,不需要额外步骤在内存中定位代码。

作为库的作者,可以决定提供静态库,还是共享库,也可以同时发布两个版本,将这个决定留给使用库的开发者。既然在积累知识,就来提供两个版本的库。

calc\_test目标包含了单元测试以验证库的核心功能,它将使用静态库。虽然是从相同的对象文件构建两种类型的库,但无论使用哪种类型的库进行测试都可以接受,它们的功能相同。与calc\_console\_static目标相关的控制台应用程序将使用共享库。此目标还链接了一个外部依赖,即Arthur Sonzogni的Functional Terminal (X) User Interface (FTXUI)库(在扩展阅读部分中有GitHub项目的链接)。

最后两个目标,calc\_console和calc\_console\_test,旨在解决测试可执行文件中的常见问题:测试框架和可执行文件提供的多个入口点之间的冲突。为此,可将main()函数隔离到一个引导目标calc\_console中,该目标仅仅调用calc\_console\_static中的主要函数。

理解了必要的目标及其相互关系之后,下一步是通过适当的文件和目录来组织项目的结构。

\mySubsubsection{15.3.2.}{项目文件结构}

项目由两个关键元素组成:calc库和calc\_console可执行文件。为了有效地组织项目,将采用以下目录结构:

\begin{itemize}
\item
src 包含所有发布的目标的源代码和库头文件。

\item
test 包含上述库和可执行文件的测试。

\item
cmake 包含用于构建和安装项目的CMake辅助模块和文件。

\item
根目录包含顶级配置和文档文件。
\end{itemize}

这一结构(如图15.3所示)确保了职责的清晰划分,便于更轻松地导航和维护项目:

\myGraphic{0.8}{content/chapter15/images/3.png}{图 15.3: 项目的目录结构}

下面是每个主要目录中的文件完整列表:

% Please add the following required packages to your document preamble:
% \usepackage{longtable}
% Note: It may be necessary to compile the document several times to get a multi-page table to line up properly
\begin{longtable}{|l|l|}
\hline
\textbf{根目录} &
\textbf{./test} \\ \hline
\endfirsthead
%
\endhead
%
CHANGELOG &
CMakeLists.txt \\ \hline
\textbf{CMakeLists.txt} &
\textbf{calc/CMakeLists.txt} \\ \hline
\begin{tabular}[c]{@{}l@{}}INSTALL\\ LICENSER\\ EADME.md\end{tabular} &
\begin{tabular}[c]{@{}l@{}}calc/calc\_test.cpp\\ calc\_console/CMakeLists.txt\\ calc\_console/tui\_test.cpp\end{tabular} \\ \hline
\textbf{./src} &
\textbf{./cmake} \\ \hline
\begin{tabular}[c]{@{}l@{}}CMakeLists.txt\\ calc/CMakeLists.txt\\ calc/CalcConfig.cmake\\ calc/basic.cpp\\ calc/include/calc/basic.h\\ calc\_console/CMakeLists.txt\\ calc\_console/bootstrap.cpp\\ calc\_console/include/tui.h\\ calc\_console/tui.cpp\end{tabular} &
\begin{tabular}[c]{@{}l@{}}BuildInfo.cmake\\ Coverage.cmake\\ CppCheck.cmake\\ Doxygen.cmake\\ Format.cmake\\ GetFTXUI.cmake\\ Packaging.cmake\\ Memcheck.cmake\\ NoInSourceBuilds.cmake\\ Testing.cmake\\ buildinfo.h.in\\ doxygen\_extra\_headers\end{tabular} \\ \hline
\end{longtable}

\begin{center}
表 15.1: 项目的文件结构
\end{center}

看起来CMake引入了相当大的开销,初始阶段cmake目录包含的内容比实际业务代码还要多,但随着项目的扩展,这种状况将会改变。建立干净且有组织的项目结构需要较大的初始努力,但请放心,这种投资在未来将带来显著的好处。

我们将在整个章节中详细介绍表15.1中列出的所有文件,以及其作用和在项目中的角色。这将分为四个步骤:构建、测试、安装和提供文档。
















































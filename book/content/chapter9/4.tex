本章为提供了使用CMake的查找模块识别系统安装的软件包的知识,以及如何利用随库提供的配置文件。对于不支持CMake但包含.pc文件的旧库,可以使用PkgConfig工具和CMake内置的FindPkgConfig查找模块。

还探讨了FetchContent模块的功能。这个模块允许在配置CMake时从各种来源下载依赖项,同时首先扫描系统,从而避免不必要的下载。我们提到了这些模块的历史背景,并讨论了在特殊情况下使用ExternalProject模块的选项。CMake设计为通过讨论的方法定位库时自动生成构建目标,这为过程增加了一层便利和优雅。

有了这个基础,就可以将标准库整合到项目中了。

下一章中,我们将学习如何使用C++20模块,在较小规模上提供可重用的代码。
您的解决方案的大小并不重要;随着它的增长,您可能会选择依赖其他项目。避免创建和维护样板代码的工作至关重要。这样可以为真正重要的事情腾出时间:业务逻辑。外部依赖项有多种用途。它们提供框架和功能,解决复杂问题,并在构建和确保代码质量中发挥关键作用。这些依赖项各不相同,从专业的编译器如Protocol Buffers(Protobuf)到测试框架如Google Test。

在使用开源项目或内部代码时,高效地管理外部依赖项是必不可少的。如果手动完成这项工作,将需要大量的设置时间和持续的支持。幸运的是,CMake擅长处理各种依赖管理方法,同时紧跟行业标准。

我们首先将学习如何识别和利用宿主系统上已经存在的依赖项,从而避免不必要的下载和延长的编译时间。这项任务相对简单,因为许多软件包要么与CMake兼容,要么直接由CMake支持。我们还将探讨如何指导CMake定位和包含那些缺乏这种本地支持的依赖项。对于遗留软件包,在特定情况下,采用另一种方法可能更有益:我们可以使用曾经流行的pkg-config工具来处理更繁琐的任务。

此外,我们还将深入探讨如何管理在线可用但尚未安装在系统上的依赖项。我们将研究如何从HTTP服务器、Git和其他类型的仓库中获取这些依赖项。我们还将讨论如何选择最佳方法:首先在系统中搜索,如果未找到该软件包,再转而获取。最后,我们将回顾一种在特殊情况下可能适用的下载外部项目的旧技术。

在本章中,将包含以下内容:

\begin{itemize}
\item
使用已安装的依赖项

\item
使用系统中不存在的依赖项
\end{itemize}


















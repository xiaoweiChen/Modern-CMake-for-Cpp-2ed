Simple compilation scenarios are usually handled by a default configuration of a toolchain or just provided out of the box by an integrated development environment (IDE). However, in a professional setting, business needs often call for something more advanced. It could be a requirement for higher performance, smaller binaries, more portability, automated testing, or extensive debugging capabilities – you name it. Managing all of these in a coherent, future-proof way quickly becomes a complex, tangled mess (especially when there are multiple platforms to support).

The process of compilation is often not explained well enough in books on C++ (in-depth subjects such as virtual base classes seem to be more interesting). In this chapter, we’ll fix that by going through different aspects of compilation: we’ll discover how compilation works, what its internal stages are, and how they affect the binary output.

After that, we will focus on the prerequisites – we’ll discuss what commands we can use to finetune the compilation process, how to require specific features from a compiler, and how to correctly instruct the compiler on which input files to process.

Then, we’ll focus on the first stage of compilation – the preprocessor. We’ll be providing paths for included headers, and we’ll study how to plug in variables from CMake and the build environment with preprocessor definitions. We’ll cover the most interesting use cases and learn how to expose CMake variables so they can be accessed from C++ code.

Right after that, we’ll talk about the optimizer and how different flags can affect performance. We’ll also discuss the costs of optimization, specifically how it affects the debuggability of produced binaries, and what to do if that isn’t desired.

Lastly, we’ll explain how to manage the compilation process in terms of reducing the compilation time by using precompiled headers and unity builds. We’ll learn how to debug the build process and find any mistakes we might’ve made.

In this chapter, we’re going to cover the following main topics:

\begin{itemize}
\item
The basics of compilation

\item
Configuring the preprocessor

\item
Configuring the optimizer

\item
Managing the process of compilation
\end{itemize}







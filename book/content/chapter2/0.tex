在CMake语言中编写代码比预期的要棘手。当你第一次阅读CMake列表文件时,你可能会觉得里面的语言如此简单,以至于可以不需要任何理论就能实践。然后你可能会试图进行更改并尝试代码,而没有彻底了解它实际上是如何工作的。我不会责怪你。我们程序员通常都很忙,而且与构建相关的问题通常不是那种让人愿意投入大量时间的事情。为了快速完成,我们倾向于基于直觉进行更改,希望它们可能会奏效。这种解决技术问题的方法被称为巫毒编程。

CMake语言看起来微不足道:在我们引入了小小的扩展、修复、黑客或单行代码之后,我们突然发现有些东西不工作了。通常,用于调试的时间超过了理解主题本身所需的时间。幸运的是,这不会是我们的命运,因为这一章节涵盖了实践中使用CMake语言所需的大部分关键知识。

在本章中,我们不仅将学习CMake语言的基本构建块——注释、命令、变量和控制结构——还将了解必要的背景知识,并遵循最新实践尝试示例。

CMake让你处于一个独特的位置。一方面,你扮演着构建工程师的角色,必须全面掌握编译器、平台及其所有相关方面。另一方面,你是一个编写生成构建系统的代码的开发者。编写高质量的代码是一项具有挑战性的任务,需要多方面的方法。代码不仅要有功能性和可读性,还应该易于分析、扩展和维护。

最后,我们将介绍CMake中最实用和最常用的命令。那些也经常使用,但程度不如前面的命令将在附录“杂项命令”(字符串、列表、文件和数学命令的参考指南)中介绍。

在本章中,我们将包括以下主要主题:

\begin{itemize}
\item
CMake语言语法的基础

\item
变量操作

\item
使用列表

\item
理解CMake中的控制结构

\item
探索常用的命令
\end{itemize}



































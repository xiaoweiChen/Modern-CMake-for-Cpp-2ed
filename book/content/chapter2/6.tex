
CMake 提供了许多脚本命令,允许您处理变量和环境。其中一些在附录中已经详细介绍了,例如 list()、string() 和 file()。其他如 find\_file()、find\_package() 和 find\_path(),更适合放在讨论它们各自主题的章节中。在本节中,我们将简要概述在大多数情况下都有用的常见命令:

\begin{itemize}
\item
message()

\item
include()

\item
include\_guard()

\item
file()

\item
execute\_process()
\end{itemize}

让我们开始吧。

\mySubsubsection{2.6.1}{message()}


我们已经知道并喜爱我们可靠的 message() 命令,它将文本打印到标准输出。然而,它的功能远不止于此。通过提供一个 MODE 参数,您可以自定义命令的行为,如下所示:message(<MODE> "text to print")。

可用的模式如下:

\begin{itemize}
\item
FATAL\_ERROR: 停止处理和生成。

\item
SEND\_ERROR: 继续处理但跳过生成。

\item
WARNING:继续处理。

\item
AUTHOR\_WARNING: CMake 警告。继续处理。

\item
DEPRECATION: 如果启用了 CMAKE\_ERROR\_DEPRECATED 或 CMAKE\_WARN\_DEPRECATED 变量,则相应地工作。

\item
NOTICE 或省略模式(默认): 打印消息到 stderr 以吸引用户的注意。

\item
STATUS: 继续处理,推荐用于向用户显示的主要消息。

\item
VERBOSE: 继续处理,应用于更详细的信息,通常不是非常必要。

\item
DEBUG: 继续处理,应包含任何可能在项目出现问题时有帮助的细微细节。

\item
TRACE: 继续处理,建议在项目开发期间打印消息。通常,这类消息会在发布项目之前被移除。
\end{itemize}

选择正确的模式需要额外的工作,但它可以通过基于严重性(自 3.21 版本起)给输出文本着色来节省调试时间,甚至在声明不可恢复的错误后停止执行:

\filename{ch02/10-useful/message\_error.cmake}

\begin{cmake}
message(FATAL_ERROR "Stop processing")
message("This won't be printed.")
\end{cmake}

消息将根据当前的日志级别(默认为 STATUS)打印。我们在上一章的调试和跟踪选项部分讨论了如何更改此设置。

在第 1 章《CMake 的第一步》中,我提到了使用 CMAKE\_MESSAGE\_CONTEXT 进行调试,现在是深入研究它的时候了。在此期间,我们获得了这个主题三个关键部分的见解:列表、作用域和函数。

在复杂的调试场景中,指出消息发生的上下文可能非常有用。考虑以下输出,其中在 foo 函数中打印的消息有适当的前缀:

\begin{shell}
$ cmake -P message_context.cmake --log-context
[top] Before `foo`
[top.foo] foo message
[top] After `foo`
\end{shell}

以下是它的工作原理:

\filename{ch02/10-useful/message\_context.cmake}

\begin{cmake}
function(foo)
    list(APPEND CMAKE_MESSAGE_CONTEXT "foo")
    message("foo message")
endfunction()

list(APPEND CMAKE_MESSAGE_CONTEXT "top")
message("Before `foo`")
foo()
message("After `foo`")
\end{cmake}

让我们来分解一下:

\begin{enumerate}
\item
首先,我们将 top 追加到上下文跟踪变量 CMAKE\_MESSAGE\_CONTEXT,然后打印初始的“在 'foo' 之前”消息,匹配的前缀 [top] 将被添加到输出中。

\item
接下来,进入 foo() 函数时,我们在属于该函数的列表后追加一个名为 foo 的新上下文,并输出另一个消息,该消息在输出中显示扩展的 [top.foo] 前缀。

\item
最后,在函数执行完成后,我们打印“在 'foo' 之后”消息。消息以原始的 [foo] 作用域打印。为什么?因为变量作用域规则:更改的 CMAKE\_MESSAGE\_CONTEXT 变量只在函数作用域结束前有效,然后恢复为原始未更改的版本
\end{enumerate}

message() 的另一个技巧是向 CMAKE\_MESSAGE\_INDENT 列表添加缩进(与 CMAKE\_MESSAGE\_CONTEXT 完全相同的方式):

\begin{cmake}
list(APPEND CMAKE_MESSAGE_INDENT " ")
message("Before `foo`")
foo()
message("After `foo`")
\end{cmake}

然后,我们的脚本输出可以看起来更简单:

\begin{shell}
Before `foo`
    foo message
After `foo`
\end{shell}

由于CMake没有提供任何带有断点或其他工具的真正调试器,所以当事情没有完全按计划进行时,生成干净的日志消息的能力非常方便。

\mySubsubsection{2.6.2}{include()}

将代码分割成不同的文件以保持有序和分离是非常有用的。然后,我们可以通过调用 include() 从我们的父列表文件引用它们,如下所示:

\begin{shell}
include(<file|module> [OPTIONAL] [RESULT_VARIABLE <var>])
\end{shell}

如果我们提供一个文件名(带有 .cmake 扩展名的路径),CMake 将尝试打开并执行它。

请注意,不会创建嵌套的独立变量作用域,因此该文件中对变量的任何更改都会影响调用作用域。

如果文件不存在,CMake 将报错,除非我们使用 OPTIONAL 关键字指定它是可选的。当我们需要知道 include() 是否成功时,可以提供 RESULT\_VARIABLE 关键字以及变量的名称。在成功时,它将被填充为包含文件的完整路径,在失败时则为 NOTFOUND。

在脚本模式下运行时,任何相对路径都将从当前工作目录解析。要强制相对于脚本本身进行搜索,请提供绝对路径:

\begin{cmake}
include("${CMAKE_CURRENT_LIST_DIR}/<filename>.cmake")
\end{cmake}

如果我们不提供路径,但提供了模块的名称(不带 .cmake 或其他),CMake 将尝试找到一个模块并包含它。CMake 将在 CMAKE\_MODULE\_PATH 中搜索名为 <模块>.cmake 的文件,然后是在 CMake 模块目录中。

当 CMake 遍历源树并包含不同的列表文件时,将设置以下变量: CMAKE\_CURRENT\_LIST\_DIR, CMAKE\_CURRENT\_LIST\_FILE, CMAKE\_PARENT\_LIST\_FILE和CMAKE\_CURRENT\_LIST\_LINE。

\mySubsubsection{2.6.3}{include\_guard()}

当我们包含具有副作用的文件时,我们可能希望限制它们,以便它们只被包含一次。这时 include\_guard([DIRECTORY|GLOBAL]) 就派上用场了。

将 include\_guard() 放在包含文件的顶部。当 CMake 第一次遇到它时,它将在当前作用域中记录这一事实。如果文件再次被包含(可能是因为我们无法控制项目中的所有文件),它就不会再被进一步处理。

如果我们希望防止在不会共享变量的不相关函数作用域中包含,我们应该提供 DIRECTORY 或 GLOBAL 参数。顾名思义,DIRECTORY 关键字将在当前目录及其以下应用保护,而 GLOBAL 关键字将把保护应用于整个构建。


\mySubsubsection{2.6.4}{file()}

为了让您了解可以在 CMake 脚本中执行的操作,让我们快速看一下文件操作命令的最有用的变体:

\begin{shell}
file(READ <filename> <out-var> [...])
file({WRITE | APPEND} <filename> <content>...)
file(DOWNLOAD <url> [<file>] [...])
\end{shell}

简而言之,file() 命令可以让您以系统无关的方式读取、写入和传输文件,以及与文件系统、文件锁、路径和存档等交互。有关更多详细信息,请参阅附录。

\mySubsubsection{2.6.5}{execute\_process()}

在需要的时候,您将不得不求助于使用系统中的工具(毕竟,CMake 主要是一个构建系统生成器)。CMake 提供了一个命令以实现此目的:您可以使用 execute\_process() 来运行其他进程并收集它们的输出。这个命令非常适合脚本使用,也可以在项目中使用,但它仅在配置阶段有效。

以下是该命令的一般形式:

\begin{shell}
execute_process(COMMAND <cmd1> [<arguments>]... [OPTIONS])
\end{shell}

CMake 将使用操作系统的 API 来创建一个子进程(因此,像 \&\&, || 和 > 这样的 shell 操作符将不起作用)。然而,您仍然可以通过多次提供 COMMAND 参数来链接命令,并将一个命令的输出传递给另一个。

可选地,您可以使用 TIMEOUT 参数来在进程未在所需限制内完成任务时终止该进程,并且可以根据需要设置 WORKING\_DIRECTORY 。

所有任务的退出代码可以通过提供 RESULTS\_VARIABLE 参数来收集到一个列表中。如果您只对最后执行的命令的结果感兴趣,请使用单数形式:RESULT\_VARIABLE 。

为了收集输出,CMake 提供了两个参数:OUTPUT\_VARIABLE 和 ERROR\_VARIABLE(它们的用法类似)。如果您希望合并 stdout 和 stderr,请为这两个参数使用相同的变量。

请记住,当为其他用户编写项目时,您应该确保您计划使用的命令在您声称支持的平台上是可用的。









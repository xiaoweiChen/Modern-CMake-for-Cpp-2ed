
为了存储一个列表,CMake会将所有元素连接成一个字符串,使用分号(;)作为分隔符:a;list;of;5;elements。可以在元素中使用反斜杠来转义分号:a\verb|\|;single\verb|\|;element。


要创建一个列表,我们可以使用set()命令:

\begin{cmake}
set(myList a list of five elements)
\end{cmake}

由于列表的存储方式,以下命令将产生完全相同的效果:

\begin{cmake}
set(myList "a;list;of;five;elements")
set(myList a list "of;five;elements")
\end{cmake}

CMake会自动在未加引号的参数中解包列表。通过传递未加引号的myList引用,我们实际上向命令发送了更多的参数:

\begin{cmake}
message("the list is:" ${myList})
\end{cmake}

message()命令将接收六个参数:“the list is:”, “a”, “list”, “of”, “five” 和 “elements”。这可能会产生意想不到的后果,因为输出将不带额外的空格打印参数:

\begin{shell}
the list is:alistoffiveelements
\end{shell}

这是一个非常简单的机制,应该小心使用。

CMake提供了一个list()命令,该命令提供了多种子命令来读取、搜索、修改和排序列表。以下是一个简短的摘要:

\begin{cmake}
list(LENGTH <list> <out-var>)
list(GET <list> <element index> [<index> ...] <out-var>)
list(JOIN <list> <glue> <out-var>)
list(SUBLIST <list> <begin> <length> <out-var>)
list(FIND <list> <value> <out-var>)
list(APPEND <list> [<element>...])
list(FILTER <list> {INCLUDE | EXCLUDE} REGEX <regex>)
list(INSERT <list> <index> [<element>...])
list(POP_BACK <list> [<out-var>...])
list(POP_FRONT <list> [<out-var>...])
list(PREPEND <list> [<element>...])
list(REMOVE_ITEM <list> <value>...)
list(REMOVE_AT <list> <index>...)
list(REMOVE_DUPLICATES <list>)
list(TRANSFORM <list> <ACTION> [...])
list(REVERSE <list>)
list(SORT <list> [...])
\end{cmake}

大多数时候,我们的项目中并不真正需要使用列表。然而,如果发现自己处于那种罕见的情况,这个概念会很方便,可以在附录中找到list()命令的更深入的参考资料。

现在,知道了如何处理各种类型的列表和变量,让我们将焦点转移到控制执行流程上,并了解CMake中可用的控制结构。

















想象一下,在一条生产线上,一台机器在钢板 上打孔。这些孔需要特定的大小和形状,以便为成品安装螺栓。生产线的设计者会设置好机器,测试这些孔,然后继续下一步。最终,某些事情会发生变化:钢材可能更厚,工人可能调整了孔的大小,或者因为设计变更需要打更多的孔。一个聪明的设计者会在关键点安装质量控制检查,以确保产品符合规格。孔是如何形成的并不重要:钻孔、冲孔还是激光切割,它们都必须满足某些要求。

同样的原则也适用于软件开发。很难预测哪些代码能够多年保持稳定,哪些将经历多次修订。随着软件功能的扩展,我们必须确保不会无意中破坏已有的东西。我们也会犯错误。即使是最优秀的程序员也无法预见每一次更改的所有影响。开发者经常要处理他们最初并未编写的代码,可能并不了解所有背后的假设。他们会阅读代码,形成心理模型,进行更改,并希望一切顺利。当这种方式不奏效时,修复错误可能需要数小时或数天的时间,并且会对产品和用户产生负面影响。

有时候,你会遇到难以理解的代码。你甚至可能开始责怪别人造成了混乱,结果却发现是自己造成的。这种情况通常发生在编写代码时过于匆忙,没有完全理解问题的情况下。

作为开发者,我们不仅受到项目截止日期或有限预算的压力;有时候我们还需要在夜间醒来修复关键问题。令人惊讶的是,一些不那么明显的错误是如何在代码审查中溜掉的。

自动化测试可以预防大多数这些问题。这些测试是代码片段,用于验证另一段代码的行为是否正确。顾名思义,每当有人进行更改时,这些测试会自动运行,通常作为构建过程的一部分。它们通常被添加为一个步骤,以确保在将代码合并到仓库之前保证代码质量。

你可能会为了节省时间而跳过创建自动化测试,但这将是一个代价高昂的错误。正如史蒂文·赖特所说:“经验是在你真正需要之后才获得的东西。”除非你正在编写一次性脚本或进行实验,否则不要跳过测试。你可能会因为精心编写的代码不断测试失败而感到沮丧。但请记住,一个失败的测试意味着你刚刚避免在生产环境中引入一个重大问题。现在花在测试上的时间将节省你以后在修复错误上的时间——并且让你晚上能睡得更香。测试并不像你想象的那么难以添加和维护。